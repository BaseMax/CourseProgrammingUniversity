\documentclass[a4paper,12pt]{article}
\usepackage{amsmath}
\usepackage{hyperref}
\usepackage{xepersian}
\settextfont[Scale = 1.0 ,
BoldFont = *Bd ,
ItalicFont = *It ,
BoldItalicFont = *BdIt ,
Extension = .ttf
]{XB Niloofar}
%\settextfont[Scale=1.2]{XB Niloofar}
\setlatintextfont[Scale=1.1]{Times New Roman}
\linespread{1.5}

\begin{document}
	\title{تمرین‌های درس برنامه‌نویسی}
	\author{دانشگاه کاشان}
	\date{}
	\maketitle
	\thispagestyle{empty}
	
	\begin{center}
		\textbf{درس: برنامه‌نویسی مقدماتی و پیشرفته}\\
		مدرس: سیدعلی محمدیه\\
    ایمیل: \href{mailto:alim@kashanu.ac.ir}{alim@kashanu.ac.ir}\\
    گیت‌هاب: \url{https://github.com/basemax}\\
		ترم: بهار 1403
	\end{center}
	
	
	\newpage
	\section*{سری اول تمرین‌ها}
	
	\begin{enumerate}
		
		    \item محاسبه مساحت دایره: شعاع دایره را از کاربر دریافت کنید و مساحت دایره را با فرمول آن حساب کنید. عدد پی را همان ۳.۱۴ در نظر بگیرید.
		\[
		\text{مساحت} = \pi r^2
		\]
		
		\item محاسبه محیط دایره: شعاع دایره را از کاربر دریافت کنید و محیط دایره را با فرمول آن محاسبه و چاپ کنید.
		\[
		\text{محیط} = 2\pi r
		\]
		
		\item تبدیل سانتی‌متر به متر و کیلومتر:‌ از کاربر یک عدد به سانتی‌متر دریافت کنید و آن را به متر و کیلومتر تبدیل و چاپ کنید.
		\[
		\text{متر} = \frac{\text{سانتی‌متر}}{100}, \quad \text{کیلومتر} = \frac{\text{سانتی‌متر}}{100000}
		\]
		
		\item محاسبه حجم مکعب: طول ضلع مکعب را از کاربر بگیرید و حجم آن را با فرمول \(a^3\) محاسبه و چاپ کنید.
		\[
		\text{حجم} = a^3
		\]
		
		\item محاسبه مساحت مثلث: قاعده و ارتفاع مثلث را از کاربر دریافت کنید و مساحت آن را با فرمول آن محاسبه کنید.
		\[
		\text{مساحت} = \frac{1}{2} \times \text{قاعده} \times \text{ارتفاع}
		\]
		
		\item تبدیل دما از فارنهایت به سانتی‌گراد: دما به فارنهایت را از کاربر بگیرید و با استفاده از فرمول آن به سانتی‌گراد تبدیل و چاپ کنید.
		\[
		\text{سانتی‌گراد} = \frac{5}{9} \times (\text{فارنهایت} - 32)
		\]
		
		\item محاسبه مساحت مستطیل: طول و عرض مستطیل را از کاربر دریافت کنید و مساحت آن را محاسبه و چاپ کنید.
		\[
		\text{مساحت} = \text{طول} \times \text{عرض}
		\]
		
		\item محاسبه حجم کره: شعاع کره را از کاربر دریافت کرده و حجم کره را با فرمول آن محاسبه و چاپ کنید.
		\[
		\text{حجم} = \frac{4}{3} \pi r^3
		\]
		
		\item تبدیل کیلوگرم به گرم و میلی‌گرم: مقدار وزن به کیلوگرم را از کاربر بگیرید و آن را به گرم و میلی‌گرم تبدیل و چاپ کنید.
		\[
		\text{گرم} = \text{کیلوگرم} \times 1000, \quad \text{میلی‌گرم} = \text{کیلوگرم} \times 1000000
		\]
		
		\item محاسبه انرژی جنبشی جسم: جرم و سرعت جسم را از کاربر دریافت کرده و انرژی جنبشی را با فرمول آن محاسبه و چاپ کنید.
		\[
		\text{انرژی جنبشی} = \frac{1}{2} m v^2
		\]
		
		\item محاسبه زمان سقوط آزاد: ارتفاع را از کاربر بگیرید و زمان سقوط آزاد را با فرمول آن که در آن \(g=9.8\) است محاسبه کنید.
		\[
		t = \sqrt{\frac{2h}{g}}
		\]
		
		\item تبدیل لیتر به میلی‌لیتر: حجم را به لیتر از کاربر دریافت کرده و به میلی‌لیتر تبدیل و چاپ کنید.
		\[
		\text{میلی‌لیتر} = \text{لیتر} \times 1000
		\]
		
		\item محاسبه مجموع و میانگین سه عدد: از کاربر سه عدد بگیرید، مجموع و میانگین آن‌ها را محاسبه و چاپ کنید.
		\[
		\text{مجموع} = a + b + c, \quad \text{میانگین} = \frac{a + b + c}{3}
		\]
		
		\item محاسبه توان دو عدد: از کاربر پایه و توان را بگیرید و عدد پایه را به توان مورد نظر برسانید و نتیجه را چاپ کنید.
		\[
		\text{نتیجه} = \text{پایه}^{\text{توان}}
		\]
		
		\item تبدیل سال به ماه، روز و ساعت: تعداد سال‌ها را از کاربر بگیرید و آن را به ماه، روز و ساعت تبدیل و چاپ کنید. فرض کنید هر سال ۳۶۵ روز باشد.
		\[
		\text{ماه} = \text{سال} \times 12, \quad \text{روز} = \text{سال} \times 365, \quad \text{ساعت} = \text{روز} \times 24
		\]
		
		\item محاسبه چگالی یک ماده: جرم و حجم ماده را از کاربر دریافت کرده و چگالی را با فرمول
		\[
		\text{چگالی} = \frac{\text{جرم}}{\text{حجم}}
		\]
		محاسبه و نمایش دهید.
		
		\item محاسبه مساحت ذوزنقه: طول اضلاع بالا و پایین و ارتفاع ذوزنقه را از کاربر بگیرید و مساحت را با فرمول
		\[
		\text{مساحت} = \frac{(a + b) \times h}{2}
		\]
		محاسبه و چاپ کنید.
		
		\item محاسبه تعداد مولکول‌ها در جرم مشخصی از ماده: جرم ماده و جرم مولی آن را از کاربر دریافت کنید و تعداد مولکول‌ها را با فرمول
		\[
		\text{تعداد مولکول‌ها} = \frac{\text{جرم}}{\text{جرم مولی}}
		\]
		محاسبه و چاپ کنید.
		
		\item تبدیل اینچ به سانتی‌متر: اندازه‌ای به اینچ را از کاربر دریافت کنید و آن را به سانتی‌متر تبدیل و نمایش دهید.
		\[
		\text{سانتی‌متر} = \text{اینچ} \times 2.54
		\]
		
		\item محاسبه مقدار سود بانکی: مبلغ اولیه، نرخ سود سالانه و تعداد سال‌ها را از کاربر دریافت کنید و سود نهایی را با فرمول آن محاسبه و چاپ کنید.
		\[
		\text{سود} = P \times r \times t
		\]
				
	\end{enumerate}
	
	\newpage
	\section*{سری دوم تمرین‌ها}
	\begin{enumerate}
		
			\item برنامه‌ای بنویسید که نام دانشجو را از ورودی بگیرد و آن را ۵ بار پشت سر هم چاپ کند.
		\item برنامه‌ای بنویسید که دو عدد از ورودی دریافت کند و حاصل جمع، تفریق، ضرب، تقسیم، و توان آن‌ها را محاسبه و چاپ کند.
		\[ 
		\text{جمع} = a + b, \quad 
		\text{تفریق} = a - b, \quad 
		\text{ضرب} = a \cdot b, \quad 
		\text{تقسیم} = \frac{a}{b}, \quad 
		\text{توان} = a^b
		\]
		\item برنامه‌ای بنویسید که عددی را از کاربر دریافت کرده و بررسی کند آیا آن عدد زوج است یا فرد.
		\[ 
		\text{زوج یا فرد} = 
		\begin{cases} 
			\text{زوج} & \text{اگر } n \bmod 2 = 0 \\ 
			\text{فرد} & \text{اگر } n \bmod 2 \neq 0 
		\end{cases} 
		\]
		\item برنامه‌ای بنویسید که نام کاربر را از ورودی بگیرد و در صورتی که تعداد حروف نام او بیشتر از ۵ حرف باشد، پیغام "نام طولانی است" را نمایش دهد، در غیر این صورت پیغام "نام کوتاه است" نمایش دهد.
		\item برنامه‌ای بنویسید که عددی را از کاربر بگیرد و سپس مربع آن عدد را چاپ کند.
		\[ 
		\text{مربع} = n^2 
		\]
		\item برنامه‌ای بنویسید که دو عدد از ورودی دریافت کرده و بررسی کند که آیا عدد اول بزرگتر یا مساوی عدد دوم است یا خیر، سپس نتیجه را چاپ کند.
		\item برنامه‌ای بنویسید که یک رشته از کاربر دریافت کند و اگر طول آن رشته ۴ یا کمتر بود، رشته را ۵ بار پشت سر هم چاپ کند.
		\item برنامه‌ای بنویسید که یک عدد صحیح از ورودی بگیرد و باقی‌مانده تقسیم آن عدد بر ۳ را محاسبه و نمایش دهد.
		\[ 
		\text{باقی‌مانده} = n \bmod 3 
		\]
		\item برنامه‌ای بنویسید که دو عدد از ورودی بگیرد و سپس مشخص کند که آیا عدد اول بزرگتر از عدد دوم است یا نه.
		\item برنامه‌ای بنویسید که دو رشته از ورودی بگیرد و بررسی کند آیا این دو رشته با هم برابر هستند یا خیر.
		\item برنامه‌ای بنویسید که عددی از ورودی بگیرد و بررسی کند که آیا آن عدد یک رقم اعشاری دارد یا خیر.
		\item برنامه‌ای بنویسید که نام کاربر را از ورودی بگیرد و اگر طول نام او بیشتر از ۳ حرف بود، اولین سه حرف آن را چاپ کند.
		\item برنامه‌ای بنویسید که یک عدد از کاربر دریافت کند و اگر عدد مثبت بود، آن را ۲ واحد اضافه کند و نمایش دهد، و اگر عدد منفی بود، آن را ۲ واحد کم کند.
		\[
		\text{عدد جدید} = 
		\begin{cases} 
			n + 2 & \text{اگر } n > 0 \\ 
			n - 2 & \text{اگر } n < 0 
		\end{cases} 
		\]
		\item برنامه‌ای بنویسید که سن کاربر را از ورودی بگیرد و اگر سن او بیشتر از ۱۸ باشد، پیغام "بالغ" و اگر کمتر از ۱۸ باشد، پیغام "نابالغ" را نمایش دهد.
		\item برنامه‌ای بنویسید که دو عدد از ورودی بگیرد و بررسی کند که آیا مجموع این دو عدد بزرگتر از ۱۰۰ است یا نه.
		\[
		\text{بررسی} = 
		\begin{cases} 
			\text{بزرگتر از ۱۰۰} & \text{اگر } a + b > 100 \\ 
			\text{کوچکتر یا مساوی ۱۰۰} & \text{اگر } a + b \leq 100 
		\end{cases} 
		\]
		\item برنامه‌ای بنویسید که عددی از ورودی بگیرد و سپس مربع، مکعب، و توان چهارم آن عدد را محاسبه و چاپ کند.
		\[
		\text{مربع} = n^2, \quad 
		\text{مکعب} = n^3, \quad 
		\text{توان چهارم} = n^4
		\]
		\item برنامه‌ای بنویسید که نام کاربر را از ورودی بگیرد و سپس سه حرف آخر آن نام را چاپ کند.
		\item برنامه‌ای بنویسید که دو عدد از ورودی بگیرد و بررسی کند آیا هر دو عدد زوج هستند یا خیر.
		\[
		\text{زوجیت هر دو} = 
		\begin{cases} 
			\text{هر دو زوج هستند} & \text{اگر } a \bmod 2 = 0 \text{ و } b \bmod 2 = 0 \\ 
			\text{حداقل یکی زوج نیست} & \text{در غیر این صورت} 
		\end{cases} 
		\]
		\item برنامه‌ای بنویسید که یک رشته و یک عدد از ورودی بگیرد و آن رشته را به تعداد برابر آن عدد چاپ کند.
		\item برنامه‌ای بنویسید که سه عدد از ورودی بگیرد و بررسی کند که آیا مجموع دو عدد اول بزرگتر از عدد سوم است یا خیر.
		\[
		\text{بررسی} = 
		\begin{cases} 
			\text{بزرگتر است} & \text{اگر } a + b > c \\ 
			\text{کوچکتر یا مساوی است} & \text{اگر } a + b \leq c 
		\end{cases} 
		\]
		
		
	\end{enumerate}
	
	
	\newpage
	\section*{سری سوم تمرین‌ها}
	\begin{enumerate}

    \item برنامه‌ای بنویسید که عددی از ورودی بگیرد و بررسی کند که آیا عدد مثبت، منفی یا صفر است، و نتیجه را چاپ کند.

\item برنامه‌ای بنویسید که یک عدد اعشاری از ورودی بگیرد و آن را به عدد صحیح تبدیل کرده و نمایش دهد.

\item برنامه‌ای بنویسید که دو عدد از کاربر دریافت کرده و اگر هر دو عدد برابر باشند، پیغام "این دو عدد برابرند" را چاپ کند.

\item برنامه‌ای بنویسید که یک رشته و یک عدد از ورودی بگیرد و اگر طول رشته کمتر از عدد بود، پیغام "رشته کوتاه است" را نمایش دهد.

\item برنامه‌ای بنویسید که یک عدد از ورودی بگیرد و بررسی کند که آیا آن عدد بر ۵ بخش‌پذیر است یا خیر.
\[
x \, \text{بر} \, 5 \, \text{بخش‌پذیر است اگر و تنها اگر: } \, x \bmod 5 = 0
\]

\item برنامه‌ای بنویسید که نام دانشجو را از ورودی بگیرد و اگر تعداد حروف نام او زوج بود، پیغام "تعداد حروف زوج است" را نمایش دهد.

\item برنامه‌ای بنویسید که عددی از ورودی بگیرد و بررسی کند که آیا عدد اول آن بین ۱ تا ۱۰ است یا خیر.

\item برنامه‌ای بنویسید که یک عدد از ورودی بگیرد و آن را با ۱۰ ضرب کرده و نتیجه را نمایش دهد.
\[
\text{نتیجه} = x \times 10
\]

\item برنامه‌ای بنویسید که عددی از ورودی بگیرد و بررسی کند که آیا آن عدد بزرگتر از ۱۰۰ است یا نه، و نتیجه را چاپ کند.

\item برنامه‌ای بنویسید که نام کاربر را از ورودی بگیرد و اولین حرف نام او را چاپ کند.

\item برنامه‌ای بنویسید که دو عدد از ورودی بگیرد و بررسی کند که آیا حاصل‌جمع آن‌ها عددی فرد است یا زوج.
\[
x = a + b, \quad x \, \text{زوج است اگر: } x \bmod 2 = 0
\]

\item برنامه‌ای بنویسید که سه عدد از ورودی بگیرد و بزرگترین عدد را نمایش دهد.
\[
\text{بزرگترین عدد} = \max(a, b, c)
\]

\item برنامه‌ای بنویسید که نام کاربر را از ورودی بگیرد و اگر نام او شامل حرف "ا" بود، پیغام "حرف ا در نام شما وجود دارد" را چاپ کند.

\item برنامه‌ای بنویسید که یک عدد صحیح از ورودی بگیرد و اگر عدد فرد بود، آن را به عدد زوج بعدی تبدیل کند و نمایش دهد.
\[
x_{\text{جدید}} = x + 1 \quad \text{اگر } x \bmod 2 \neq 0
\]

\item برنامه‌ای بنویسید که دو عدد از ورودی بگیرد و بررسی کند که آیا هر دو عدد هم‌علامت (مثبت یا منفی) هستند یا خیر.

\item برنامه‌ای بنویسید که عددی از ورودی بگیرد و بررسی کند که آیا آن عدد بر ۳ یا ۵ بخش‌پذیر است یا خیر.
\[
x \, \text{بر ۳ یا ۵ بخش‌پذیر است اگر: } x \bmod 3 = 0 \, \text{یا} \, x \bmod 5 = 0
\]

\item برنامه‌ای بنویسید که یک رشته از ورودی بگیرد و تعداد کاراکترهای آن رشته را نمایش دهد.

\item برنامه‌ای بنویسید که سن کاربر را از ورودی بگیرد و اگر بین ۱۳ تا ۱۹ باشد، پیغام "شما نوجوان هستید" را چاپ کند.

\item برنامه‌ای بنویسید که یک عدد اعشاری از ورودی بگیرد و آن را به نزدیک‌ترین عدد صحیح گرد کند و نمایش دهد.
\[
x_{\text{گرد شده}} = \text{round}(x)
\]

\item برنامه‌ای بنویسید که سه عدد از ورودی بگیرد و بررسی کند که آیا مجموع آن‌ها بین ۱۰۰ و ۲۰۰ است یا خیر.
\[
100 \leq (a + b + c) \leq 200
\]

	\end{enumerate}
	
	\newpage
	\section*{سری چهارم تمرین‌ها}
	\begin{enumerate}
    \item برنامه‌ای بنویسید که از ۱ تا ۱۰ را چاپ کند.

\item برنامه‌ای بنویسید که از عددی که کاربر وارد می‌کند تا ۱ شمارش معکوس انجام دهد.

\item برنامه‌ای بنویسید که مجموع اعداد از ۱ تا ۲۰ را محاسبه کرده و نمایش دهد.
\[
S = \sum_{i=1}^{20} i
\]

\item برنامه‌ای بنویسید که تمامی اعداد فرد از ۱ تا ۵۰ را چاپ کند.

\item برنامه‌ای بنویسید که تعداد ارقام یک عدد را از کاربر بگیرد و تمام ارقام آن را چاپ کند.

\item برنامه‌ای بنویسید که مجموع اعداد فرد از ۱ تا ۱۰۰ را محاسبه کرده و نمایش دهد.
\[
S = \sum_{i=1, \, i \, \text{فرد}}^{100} i
\]

\item برنامه‌ای بنویسید که اعداد از ۲ تا ۲۰ را که بر ۳ بخش‌پذیر هستند، چاپ کند.

\item برنامه‌ای بنویسید که کاربر را تا زمانی که عددی منفی وارد کند، از او عدد بخواهد.

\item برنامه‌ای بنویسید که اعداد از ۱ تا ۵۰۰ را که بر ۴ بخش‌پذیر هستند، چاپ کند.

\item برنامه‌ای بنویسید که تعداد اعداد مثبت وارد شده توسط کاربر را تا زمانی که عدد ۰ وارد کند، محاسبه کند.

\item برنامه‌ای بنویسید که تا ۵۰ عدد اول را پیدا کرده و چاپ کند.

\item برنامه‌ای بنویسید که از کاربر سه عدد بگیرد و بزرگترین عدد را با استفاده از حلقه‌ها پیدا کند.

\item برنامه‌ای بنویسید که تمامی اعداد زوج بین ۱ تا ۵۰ را با استفاده از حلقه چاپ کند.

\item برنامه‌ای بنویسید که مجموع اعداد از ۱ تا ۵۰ را با استفاده از حلقه \texttt{while} محاسبه کند.
\[
S = \sum_{i=1}^{50} i
\]

\item برنامه‌ای بنویسید که تمام اعداد مضاعف ۷ بین ۱ تا ۱۰۰ را چاپ کند.

\item برنامه‌ای بنویسید که تعداد اعداد منفی وارد شده توسط کاربر را تا زمانی که عدد مثبت وارد کند، محاسبه کند.

\item برنامه‌ای بنویسید که اعداد ۱ تا ۱۰۰ را چاپ کند، اما برای اعداد قابل بخش‌پذیر بر ۳، کلمه "Fizz" و برای اعداد قابل بخش‌پذیر بر ۵، کلمه "Buzz" را چاپ کند.

\item برنامه‌ای بنویسید که تعداد اعداد اول از ۱ تا ۱۰۰ را با استفاده از حلقه‌ها محاسبه کند.

\item برنامه‌ای بنویسید که از کاربر بخواهد تا عددی را وارد کند و سپس برنامه تمام مضارب آن عدد تا ۱۰۰ را چاپ کند.
\[
x_i = n \cdot i \quad \text{که } x_i \leq 100
\]

\item برنامه‌ای بنویسید که از کاربر بخواهد یک عدد وارد کند و سپس تمام اعداد از ۱ تا آن عدد را که بر ۲ بخش‌پذیر نیستند را چاپ کند.

	\end{enumerate}
	
	
	
	\newpage
	\section*{سری پنجم تمرین‌ها}
	\begin{enumerate}
		
		
		    \item برنامه‌ای بنویسید که اعداد فیبوناچی تا \(n\) را چاپ کند. 
		\[
		F(n) = F(n-1) + F(n-2) \quad \text{با شرایط اولیه: } F(0) = 0, \, F(1) = 1
		\]
		
		\item برنامه‌ای بنویسید که تعداد ارقام زوج یک عدد را محاسبه کند و چاپ کند.
		\[
		\text{تعداد ارقام زوج } = \sum_{i=1}^{n} \text{اگر } d_i \mod 2 = 0
		\]
		
		\item برنامه‌ای بنویسید که معکوس یک رشته را چاپ کند.
		
		\item برنامه‌ای بنویسید که اعداد اول بین دو عدد داده شده را چاپ کند.
		
		\item برنامه‌ای بنویسید که جمع اعداد فرد بین \(n\) و \(m\) را محاسبه کند و چاپ کند.
		\[
		S = \sum_{i=n}^{m} i \quad \text{برای } i \text{ که فرد است.}
		\]
		
		\item برنامه‌ای بنویسید که تعداد ارقام فرد یک عدد را محاسبه کند و چاپ کند.
		\[
		\text{تعداد ارقام فرد } = \sum_{i=1}^{n} \text{اگر } d_i \mod 2 \neq 0
		\]
		
		\item برنامه‌ای بنویسید که عدد اول بعد از \(n\) را پیدا کند و چاپ کند.
		
		\item برنامه‌ای بنویسید که معکوس لیستی از اعداد را چاپ کند.
		
		\item برنامه‌ای بنویسید که عوامل یک عدد را شناسایی کرده و چاپ کند.
		\[
		d \quad \text{که } n \mod d = 0
		\]
		
		\item برنامه‌ای بنویسید که عددی را از کاربر بگیرد و بزرگترین عامل آن را چاپ کند.
		\[
		\text{بزرگترین عامل } = \max(d \quad \text{که } n \mod d = 0)
		\]
		
		\item برنامه‌ای بنویسید که عدد اول بعد از یک عدد داده شده را چاپ کند.
		
		\item برنامه‌ای بنویسید که تعداد ارقام متفاوت یک عدد را محاسبه کرده و چاپ کند.
		
		\item برنامه‌ای بنویسید که اولین عدد بین \(n\) و \(m\) که بر \(k\) بخش‌پذیر است را چاپ کند.
		\[
		x \quad \text{که} x \mod k = 0
		\]
		
		\item برنامه‌ای بنویسید که اعداد اول بین \(m\) و \(n\) را محاسبه کرده و چاپ کند.
		
		\item برنامه‌ای بنویسید که حد بالایی عدد اول \(k\)-ام را محاسبه کرده و چاپ کند.
		
		\item برنامه‌ای بنویسید که معکوس یک عدد را بدون استفاده از رشته تبدیل کند و چاپ کند.
		
		\item برنامه‌ای بنویسید که مجموع ارقام یک عدد را محاسبه کند و چاپ کند.
		\[
		S = \sum_{i=1}^{n} d_i \quad
		\]
		
		
		\begin{center}
			که \( d_i \) ارقام عدد \( n \) هستند.
		\end{center}
		
		
		\item برنامه‌ای بنویسید که گرمایش بدن انسان را بر اساس دمای داده شده تبدیل به فارنهایت کند.
		\[
		F = \frac{9}{5}C + 32
		\]
		
		\item برنامه‌ای بنویسید که دنباله زمانی را با استفاده از حلقه \texttt{while} بررسی کند و وضعیت آن را چاپ کند.
		
    \item 	برنامه‌ای بنویسید که عدد \(n\) را از کاربر بگیرد و مجموع ارقام آن را محاسبه و چاپ کند.
\[
S = \sum_{i=1}^{n} d_i \quad \text{که } d_i
\]

    \begin{center}
	که \( d_i \) ارقام عدد \( n \) هستند.
\end{center}

		
	\end{enumerate}
	
\end{document}
