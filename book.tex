\documentclass[a4paper,12pt]{article}
\usepackage{amsmath}
\usepackage{hyperref}
\usepackage{advdate}
\usepackage{xepersian}

\settextfont[
    Path=fonts/,
    Scale=1.0,
    BoldFont=*Bd,
    ItalicFont=*It,
    BoldItalicFont=*BdIt,
    Extension=.ttf
]{XB Niloofar}

\setlatintextfont[
    Path=fonts/,
    Scale=1.1
]{Times New Roman}

\newfontfamily\englishfont[
    Path=fonts/,
]{Times New Roman}

\linespread{1.5}

\begin{document}
	\title{یادگیری برنامه‌نویسی با تمرین}
	\author{دانشگاه کاشان}
	\date{}
	\maketitle
	\thispagestyle{empty}
	
	\begin{center}
		\textbf{درس: برنامه‌نویسی مقدماتی}\\
		مدرس: سیدعلی محمدیه\\
		ایمیل: \href{mailto:alim@kashanu.ac.ir}{alim@kashanu.ac.ir}\\
		گیت‌هاب: \url{https://github.com/basemax}\\

		\vspace{1cm}
		
	نسخه: 1\\
			ترم: بهار 1403  \\
				آخرین ویرایش: \today\\
				
		\vspace{1cm}


		\begin{LTR}
			\textbf{Seyyed Ali Mohammadiyeh}\\
			Department of Pure Mathematics, Faculty of Mathematical Sciences, University of Kashan, Kashan 87317-53153, I. R. Iran		
		\end{LTR}
		
				گروه ریاضی محض، دانشکده علوم ریاضی، دانشگاه کاشان، کاشان، ایران\\
	\end{center}
	
	
	\newpage
	\section*{مقدمه}
	
	\subsection*{اهمیت برنامه‌نویسی و کامپیوتر}
	در دنیای امروز، کامپیوترها و برنامه‌نویسی نقش حیاتی در تمامی جنبه‌های زندگی ما ایفا می‌کنند. از اپلیکیشن‌های موبایل و وب‌سایت‌ها گرفته تا سیستم‌های پیشرفته هوش مصنوعی و تحلیل داده‌ها، برنامه‌نویسی ابزار اصلی برای ایجاد و پیشرفت در این حوزه ها است. یادگیری برنامه‌نویسی نه تنها مهارتی ارزشمند برای حرفه شما است، بلکه به شما کمک می‌کند که مسائل پیچیده را تجزیه و تحلیل کرده و راه‌حل‌های نوآورانه ارائه دهید.
	
	\subsection*{چرا باید برنامه‌نویسی را یاد بگیریم؟}
	برنامه‌نویسی مهارتی است که تفکر منطقی، خلاقیت، و توانایی حل مسئله را تقویت می‌کند. با یادگیری برنامه‌نویسی، می‌توانید ایده‌های خود را به واقعیت تبدیل کرده و دنیای اطراف خود را بهتر بشناسید. این مسیر یادگیری شما را قادر می‌سازد که در هر صنعتی که علاقه دارید، از تکنولوژی بهره ببرید و به نوآوری بپردازید.
	
	\section*{به یادگیری ادامه دهید!}

	این دوره و تمرین‌های ارائه‌شده در این کتاب، تنها آغاز راه شما در دنیای برنامه‌نویسی است. یادگیری، سفری بی‌پایان است که با هر گام، افق‌های جدیدی را پیش روی شما می‌گشاید. همان‌طور که سقراط می‌گوید:

	\begin{quote}
	«من دانم که هیچ نمی‌دانم.»
	\end{quote}

	این جمله، حقیقتی عمیق را در دل خود دارد: هرچه بیشتر یاد بگیریم، بیشتر به وسعت نادانسته‌های خود پی می‌بریم. دنیای فناوری همواره در حال تغییر است و بهترین برنامه‌نویسان کسانی هستند که با اشتیاق، همواره در حال یادگیری و تطبیق با پیشرفت‌ها هستند.

	تمرین مداوم و یادگیری مستقل، کلید موفقیت شماست. همان‌طور که ریچارد فاینمن، فیزیکدان و دانشمند کامپیوتر، می‌گوید:

	\begin{quote}
	«بهترین راه برای یادگیری این است که چیزی را به کسی آموزش دهید.»
	\end{quote}

	پس سعی کنید دانسته‌های خود را با دیگران به اشتراک بگذارید، روی پروژه‌های واقعی کار کنید و از حل چالش‌ها نترسید. مشکلات، فرصتی برای رشد شما هستند. گریگوری چیتین، یکی از نظریه‌پردازان برجسته در علوم کامپیوتر، می‌گوید:

	\begin{quote}
	«خلاقیت در جایی آغاز می‌شود که دانش به پایان می‌رسد.»
	\end{quote}

	با عشق و علاقه یادگیری را ادامه دهید، زیرا این مسیر نه‌تنها به مهارت‌های شما می‌افزاید، بلکه ذهن شما را برای حل مسائل پیچیده‌تر آماده می‌کند. استیو جابز، بنیان‌گذار اپل، نیز این موضوع را به زیبایی بیان کرده است:

	\begin{quote}
	«تنها راه انجام کارهای بزرگ، داشتن عشق به آن چیزی است که انجام می‌دهید.»
	\end{quote}

	به یاد داشته باشید که مسیر یادگیری شما بی‌پایان و سرشار از فرصت است. شک نکنید که هر روزی که به یادگیری ادامه می‌دهید، یک قدم به تسلط بیشتر بر مهارت‌های خود نزدیک‌تر می‌شوید. پس به کاوش ادامه دهید، به رشد خود ایمان داشته باشید و هرگز از یادگیری باز نایستید!

		
	\subsection*{خلاصه مطالب این کتاب و دوره}
	
	در این کتاب و دوره آموزشی، شما گام‌به‌گام با اصول برنامه‌نویسی آشنا خواهید شد و مهارت‌های لازم برای شروع و پیشرفت در دنیای برنامه‌نویسی را کسب خواهید کرد. این مطالب به‌گونه‌ای طراحی شده‌اند که حتی اگر هیچ تجربه‌ای در این زمینه ندارید، بتوانید با اشتیاق و اعتماد به نفس وارد این مسیر شوید. در زیر، خلاصه‌ای از موضوعاتی که در این کتاب و دوره پوشش داده می‌شود آمده است:
	
	\begin{itemize}
		\item \textbf{مبانی برنامه‌نویسی:} یادگیری مفاهیم اصلی مانند متغیرها، انواع داده، عملگرها و ورودی و خروجی.
		\item \textbf{ساختارهای کنترلی:} آشنایی با شرط‌ها (\lr{if, else, elif}) و حلقه‌ها (\lr{for, while}) برای ایجاد منطق در برنامه‌ها.
		\item \textbf{توابع و ماژول‌ها:} تعریف و استفاده از توابع برای سازمان‌دهی کدها و بهره‌گیری از کتابخانه‌های استاندارد پایتون.
		\item \textbf{کار با فایل‌ها:} ایجاد، خواندن، ویرایش و ذخیره فایل‌های متنی و داده‌های ساختاریافته مانند \lr{JSON}.
		\item \textbf{تعامل با پایگاه‌داده:} اتصال به پایگاه‌های داده، ایجاد جداول، و انجام عملیات‌های درج، ویرایش و حذف داده‌ها.
		\item \textbf{کار با اینترنت:} ارسال درخواست‌های \lr{HTTP} و دریافت اطلاعات از منابع آنلاین با استفاده از کتابخانه \lr{Requests}.
		\item \textbf{پردازش تصویر:} آشنایی با مفاهیم اولیه پردازش تصویر با \lr{OpenCV} و پیاده‌سازی پروژه‌هایی مانند تشخیص چهره و اشیاء.
		\item \textbf{پروژه‌های کاربردی:} پیاده‌سازی پروژه‌های واقعی برای به‌کارگیری و تقویت مهارت‌های آموخته شده.
		\item \textbf{روش‌های حل مسئله:} توسعه تفکر منطقی و یادگیری تکنیک‌های حل مسئله در برنامه‌نویسی.
		\item \textbf{چالش‌های پیشرفته:} بررسی تمرین‌ها و پروژه‌هایی که شما را برای مقابله با مسائل پیچیده‌تر آماده می‌کند.
	\end{itemize}
	
	این دوره و کتاب به شما کمک می‌کنند تا مهارت‌های خود را گسترش دهید و با اعتماد به نفس وارد دنیای شگفت‌انگیز برنامه‌نویسی شوید.
	
	
	\subsection*{درباره مدرس}
	
	مدرس این دوره، علی، یکی از توسعه‌دهندگان برجسته و با تجربه در زمینه برنامه‌نویسی است. علی از کودکی به دنیای کدنویسی علاقه‌مند شد و از همان زمان به یادگیری و توسعه مهارت‌های خود پرداخت. او به عنوان یک برنامه‌نویس فعال در سطح بین‌المللی شناخته شده است و با بیش از ۳۰ شرکت در پروژه‌های متنوعی در حوزه‌های مختلف مانند وب، موبایل، هوش مصنوعی، و سیستم‌های توزیع‌شده همکاری داشته است.
	
	علی همچنین یکی از برترین مشارکت‌کنندگان گیت‌هاب به شمار می‌آید و بیش از ۱۰۰۰ مخزن متن‌باز را مدیریت می‌کند. علاقه او به متن‌باز نه تنها به توسعه ابزارها و کتابخانه‌های کاربردی کمک کرده است، بلکه زمینه‌ای برای یادگیری و همکاری برنامه‌نویسان از سراسر جهان فراهم کرده است.
	
	علی در کنار حرفه‌ای بودن در برنامه‌نویسی، به آموزش و انتقال دانش نیز علاقه‌مند است. او معتقد است که دانش باید به اشتراک گذاشته شود تا به پیشرفت جامعه کمک کند. این دوره، نتیجه تلاش‌های او برای ایجاد یک مسیر ساده و جذاب برای ورود به دنیای برنامه‌نویسی است. 
	
	در این کتاب و دوره، شما نه تنها اصول برنامه‌نویسی را یاد خواهید گرفت، بلکه به درکی عمیق از روش حل مسئله و توسعه مهارت‌های تفکر منطقی دست خواهید یافت. علی اطمینان دارد که هر کسی با تلاش و پشتکار می‌تواند به یک برنامه‌نویس موفق تبدیل شود.
	
	
	\newpage
	
	\section*{یادبود استاد بزرگوار، پروفسور علیرضا اشرفی}
	
	در آغاز این کتاب، مایلم از فرصت استفاده کرده و یاد و خاطره‌ی استاد بزرگوارم، پروفسور علیرضا اشرفی را گرامی بدارم. ایشان نه تنها در زمینه علمی‌ام به عنوان سوپروایزر در دوران تحصیلم در دانشگاه کاشان، بلکه در بسیاری از جنبه‌های زندگی‌ام تاثیرگذار بودند.
	
	پروفسور اشرفی با عمق دانش و نگاه دقیق علمی‌ خود، همواره به ما دانشجویان این امکان را می‌دادند تا نه تنها در مسائل ریاضی، بلکه در نحوه تفکر و برخورد با چالش‌ها رشد کنیم. ایشان با داشتن اخلاق حرفه‌ای و دلسوزی بی‌نظیر، به ما آموختند که علم باید با انسانیت همراه باشد. این ویژگی‌های ایشان نه تنها در تدریس، بلکه در برخوردهای روزمره‌ و مشاوره‌های علمی‌ ایشان نمایان بود.
	
	در دوران تحصیل، به‌ویژه در پروژه‌های تحقیقاتی که تحت نظارت ایشان انجام دادم، ایشان همیشه به من انگیزه می‌دادند تا بهترین خود را ارائه دهم و از هیچ تلاشی برای یادگیری و بهبود خود دریغ نکنم. ایشان نه تنها یک استاد در زمینه‌های علمی، بلکه یک الگو در زندگی حرفه‌ای و شخصی برای من و بسیاری از دانشجویان دیگر بودند.
	
	پروفسور علیرضا اشرفی تا آخرین روزهای زندگی‌ خود، در مسیر توسعه علم و آموزش متعهد بودند و سهم بزرگی در پیشرفت علمی کشور داشتند. یاد ایشان همواره در دل ما زنده خواهد ماند.
	
	
	\newpage
	\section*{تمرین‌های سری اول}
	
	\begin{enumerate}
		
		\item محاسبه مساحت دایره: شعاع دایره را از کاربر دریافت کنید و مساحت دایره را با فرمول آن حساب کنید. عدد پی را همان ۳.۱۴ در نظر بگیرید.
		\[
		\text{مساحت} = \pi r^2
		\]
		
		\item محاسبه محیط دایره: شعاع دایره را از کاربر دریافت کنید و محیط دایره را با فرمول آن محاسبه و چاپ کنید.
		\[
		\text{محیط} = 2\pi r
		\]
		
		\item تبدیل سانتی‌متر به متر و کیلومتر:‌ از کاربر یک عدد به سانتی‌متر دریافت کنید و آن را به متر و کیلومتر تبدیل و چاپ کنید.
		\[
		\text{متر} = \frac{\text{سانتی‌متر}}{100}, \quad \text{کیلومتر} = \frac{\text{سانتی‌متر}}{100000}
		\]
		
		\item محاسبه حجم مکعب: طول ضلع مکعب را از کاربر بگیرید و حجم آن را با فرمول \(a^3\) محاسبه و چاپ کنید.
		\[
		\text{حجم} = a^3
		\]
		
		\item محاسبه مساحت مثلث: قاعده و ارتفاع مثلث را از کاربر دریافت کنید و مساحت آن را با فرمول آن محاسبه کنید.
		\[
		\text{مساحت} = \frac{1}{2} \times \text{قاعده} \times \text{ارتفاع}
		\]
		
		\item تبدیل دما از فارنهایت به سانتی‌گراد: دما به فارنهایت را از کاربر بگیرید و با استفاده از فرمول آن به سانتی‌گراد تبدیل و چاپ کنید.
		\[
		\text{سانتی‌گراد} = \frac{5}{9} \times (\text{فارنهایت} - 32)
		\]
		
		\item محاسبه مساحت مستطیل: طول و عرض مستطیل را از کاربر دریافت کنید و مساحت آن را محاسبه و چاپ کنید.
		\[
		\text{مساحت} = \text{طول} \times \text{عرض}
		\]
		
		\item محاسبه حجم کره: شعاع کره را از کاربر دریافت کرده و حجم کره را با فرمول آن محاسبه و چاپ کنید.
		\[
		\text{حجم} = \frac{4}{3} \pi r^3
		\]
		
		\item تبدیل کیلوگرم به گرم و میلی‌گرم: مقدار وزن به کیلوگرم را از کاربر بگیرید و آن را به گرم و میلی‌گرم تبدیل و چاپ کنید.
		\[
		\text{گرم} = \text{کیلوگرم} \times 1000, \quad \text{میلی‌گرم} = \text{کیلوگرم} \times 1000000
		\]
		
		\item محاسبه انرژی جنبشی جسم: جرم و سرعت جسم را از کاربر دریافت کرده و انرژی جنبشی را با فرمول آن محاسبه و چاپ کنید.
		\[
		\text{انرژی جنبشی} = \frac{1}{2} m v^2
		\]
		
		\item محاسبه زمان سقوط آزاد: ارتفاع را از کاربر بگیرید و زمان سقوط آزاد را با فرمول آن که در آن \(g=9.8\) است محاسبه کنید.
		\[
		t = \sqrt{\frac{2h}{g}}
		\]
		
		\item تبدیل لیتر به میلی‌لیتر: حجم را به لیتر از کاربر دریافت کرده و به میلی‌لیتر تبدیل و چاپ کنید.
		\[
		\text{میلی‌لیتر} = \text{لیتر} \times 1000
		\]
		
		\item محاسبه مجموع و میانگین سه عدد: از کاربر سه عدد بگیرید، مجموع و میانگین آن‌ها را محاسبه و چاپ کنید.
		\[
		\text{مجموع} = a + b + c, \quad \text{میانگین} = \frac{a + b + c}{3}
		\]
		
		\item محاسبه توان دو عدد: از کاربر پایه و توان را بگیرید و عدد پایه را به توان مورد نظر برسانید و نتیجه را چاپ کنید.
		\[
		\text{نتیجه} = \text{پایه}^{\text{توان}}
		\]
		
		\item تبدیل سال به ماه، روز و ساعت: تعداد سال‌ها را از کاربر بگیرید و آن را به ماه، روز و ساعت تبدیل و چاپ کنید. فرض کنید هر سال ۳۶۵ روز باشد.
		\[
		\text{ماه} = \text{سال} \times 12, \quad \text{روز} = \text{سال} \times 365, \quad \text{ساعت} = \text{روز} \times 24
		\]
		
		\item محاسبه چگالی یک ماده: جرم و حجم ماده را از کاربر دریافت کرده و چگالی را با فرمول
		\[
		\text{چگالی} = \frac{\text{جرم}}{\text{حجم}}
		\]
		محاسبه و نمایش دهید.
		
		\item محاسبه مساحت ذوزنقه: طول اضلاع بالا و پایین و ارتفاع ذوزنقه را از کاربر بگیرید و مساحت را با فرمول
		\[
		\text{مساحت} = \frac{(a + b) \times h}{2}
		\]
		محاسبه و چاپ کنید.
		
		\item محاسبه تعداد مولکول‌ها در جرم مشخصی از ماده: جرم ماده و جرم مولی آن را از کاربر دریافت کنید و تعداد مولکول‌ها را با فرمول
		\[
		\text{تعداد مولکول‌ها} = \frac{\text{جرم}}{\text{جرم مولی}}
		\]
		محاسبه و چاپ کنید.
		
		\item تبدیل اینچ به سانتی‌متر: اندازه‌ای به اینچ را از کاربر دریافت کنید و آن را به سانتی‌متر تبدیل و نمایش دهید.
		\[
		\text{سانتی‌متر} = \text{اینچ} \times 2.54
		\]
		
		\item محاسبه مقدار سود بانکی: مبلغ اولیه، نرخ سود سالانه و تعداد سال‌ها را از کاربر دریافت کنید و سود نهایی را با فرمول آن محاسبه و چاپ کنید.
		\[
		\text{سود} = P \times r \times t
		\]
		
	\end{enumerate}
	
	\newpage
	\section*{تمرین‌های سری دوم}
	\begin{enumerate}
		
		\item برنامه‌ای بنویسید که نام دانشجو را از ورودی بگیرد و آن را ۵ بار پشت سر هم چاپ کند.
		\item برنامه‌ای بنویسید که دو عدد از ورودی دریافت کند و حاصل جمع، تفریق، ضرب، تقسیم، و توان آن‌ها را محاسبه و چاپ کند.
		\[ 
		\text{جمع} = a + b, \quad 
		\text{تفریق} = a - b, \quad 
		\text{ضرب} = a \cdot b, \quad 
		\text{تقسیم} = \frac{a}{b}, \quad 
		\text{توان} = a^b
		\]
		\item برنامه‌ای بنویسید که عددی را از کاربر دریافت کرده و بررسی کند آیا آن عدد زوج است یا فرد.
		\[ 
		\text{زوج یا فرد} = 
		\begin{cases} 
			\text{زوج} & \text{اگر } n \bmod 2 = 0 \\ 
			\text{فرد} & \text{اگر } n \bmod 2 \neq 0 
		\end{cases} 
		\]
		\item برنامه‌ای بنویسید که نام کاربر را از ورودی بگیرد و در صورتی که تعداد حروف نام او بیشتر از ۵ حرف باشد، پیغام "نام طولانی است" را نمایش دهد، در غیر این صورت پیغام "نام کوتاه است" نمایش دهد.
		\item برنامه‌ای بنویسید که عددی را از کاربر بگیرد و سپس مربع آن عدد را چاپ کند.
		\[ 
		\text{مربع} = n^2 
		\]
		\item برنامه‌ای بنویسید که دو عدد از ورودی دریافت کرده و بررسی کند که آیا عدد اول بزرگتر یا مساوی عدد دوم است یا خیر، سپس نتیجه را چاپ کند.
		\item برنامه‌ای بنویسید که یک رشته از کاربر دریافت کند و اگر طول آن رشته ۴ یا کمتر بود، رشته را ۵ بار پشت سر هم چاپ کند.
		\item برنامه‌ای بنویسید که یک عدد صحیح از ورودی بگیرد و باقی‌مانده تقسیم آن عدد بر ۳ را محاسبه و نمایش دهد.
		\[ 
		\text{باقی‌مانده} = n \bmod 3 
		\]
		\item برنامه‌ای بنویسید که دو عدد از ورودی بگیرد و سپس مشخص کند که آیا عدد اول بزرگتر از عدد دوم است یا نه.
		\item برنامه‌ای بنویسید که دو رشته از ورودی بگیرد و بررسی کند آیا این دو رشته با هم برابر هستند یا خیر.
		\item برنامه‌ای بنویسید که عددی از ورودی بگیرد و بررسی کند که آیا آن عدد یک رقم اعشاری دارد یا خیر.
		\item برنامه‌ای بنویسید که نام کاربر را از ورودی بگیرد و اگر طول نام او بیشتر از ۳ حرف بود، اولین سه حرف آن را چاپ کند.
		\item برنامه‌ای بنویسید که یک عدد از کاربر دریافت کند و اگر عدد مثبت بود، آن را ۲ واحد اضافه کند و نمایش دهد، و اگر عدد منفی بود، آن را ۲ واحد کم کند.
		\[
		\text{عدد جدید} = 
		\begin{cases} 
			n + 2 & \text{اگر } n > 0 \\ 
			n - 2 & \text{اگر } n < 0 
		\end{cases} 
		\]
		\item برنامه‌ای بنویسید که سن کاربر را از ورودی بگیرد و اگر سن او بیشتر از ۱۸ باشد، پیغام "بالغ" و اگر کمتر از ۱۸ باشد، پیغام "نابالغ" را نمایش دهد.
		\item برنامه‌ای بنویسید که دو عدد از ورودی بگیرد و بررسی کند که آیا مجموع این دو عدد بزرگتر از ۱۰۰ است یا نه.
		\[
		\text{بررسی} = 
		\begin{cases} 
			\text{بزرگتر از ۱۰۰} & \text{اگر } a + b > 100 \\ 
			\text{کوچکتر یا مساوی ۱۰۰} & \text{اگر } a + b \leq 100 
		\end{cases} 
		\]
		\item برنامه‌ای بنویسید که عددی از ورودی بگیرد و سپس مربع، مکعب، و توان چهارم آن عدد را محاسبه و چاپ کند.
		\[
		\text{مربع} = n^2, \quad 
		\text{مکعب} = n^3, \quad 
		\text{توان چهارم} = n^4
		\]
		\item برنامه‌ای بنویسید که نام کاربر را از ورودی بگیرد و سپس سه حرف آخر آن نام را چاپ کند.
		\item برنامه‌ای بنویسید که دو عدد از ورودی بگیرد و بررسی کند آیا هر دو عدد زوج هستند یا خیر.
		\[
		\text{زوجیت هر دو} = 
		\begin{cases} 
			\text{هر دو زوج هستند} & \text{اگر } a \bmod 2 = 0 \text{ و } b \bmod 2 = 0 \\ 
			\text{حداقل یکی زوج نیست} & \text{در غیر این صورت} 
		\end{cases} 
		\]
		\item برنامه‌ای بنویسید که یک رشته و یک عدد از ورودی بگیرد و آن رشته را به تعداد برابر آن عدد چاپ کند.
		\item برنامه‌ای بنویسید که سه عدد از ورودی بگیرد و بررسی کند که آیا مجموع دو عدد اول بزرگتر از عدد سوم است یا خیر.
		\[
		\text{بررسی} = 
		\begin{cases} 
			\text{بزرگتر است} & \text{اگر } a + b > c \\ 
			\text{کوچکتر یا مساوی است} & \text{اگر } a + b \leq c 
		\end{cases} 
		\]
		
		
	\end{enumerate}
	
	
	\newpage
	\section*{تمرین‌های سری سوم}
	\begin{enumerate}
		
		\item برنامه‌ای بنویسید که عددی از ورودی بگیرد و بررسی کند که آیا عدد مثبت، منفی یا صفر است، و نتیجه را چاپ کند.
		
		\item برنامه‌ای بنویسید که یک عدد اعشاری از ورودی بگیرد و آن را به عدد صحیح تبدیل کرده و نمایش دهد.
		
		\item برنامه‌ای بنویسید که دو عدد از کاربر دریافت کرده و اگر هر دو عدد برابر باشند، پیغام "این دو عدد برابرند" را چاپ کند.
		
		\item برنامه‌ای بنویسید که یک رشته و یک عدد از ورودی بگیرد و اگر طول رشته کمتر از عدد بود، پیغام "رشته کوتاه است" را نمایش دهد.
		
		\item برنامه‌ای بنویسید که یک عدد از ورودی بگیرد و بررسی کند که آیا آن عدد بر ۵ بخش‌پذیر است یا خیر.
		\[
		x \, \text{بر} \, 5 \, \text{بخش‌پذیر است اگر و تنها اگر: } \, x \bmod 5 = 0
		\]
		
		\item برنامه‌ای بنویسید که نام دانشجو را از ورودی بگیرد و اگر تعداد حروف نام او زوج بود، پیغام "تعداد حروف زوج است" را نمایش دهد.
		
		\item برنامه‌ای بنویسید که عددی از ورودی بگیرد و بررسی کند که آیا عدد اول آن بین ۱ تا ۱۰ است یا خیر.
		
		\item برنامه‌ای بنویسید که یک عدد از ورودی بگیرد و آن را با ۱۰ ضرب کرده و نتیجه را نمایش دهد.
		\[
		\text{نتیجه} = x \times 10
		\]
		
		\item برنامه‌ای بنویسید که عددی از ورودی بگیرد و بررسی کند که آیا آن عدد بزرگتر از ۱۰۰ است یا نه، و نتیجه را چاپ کند.
		
		\item برنامه‌ای بنویسید که نام کاربر را از ورودی بگیرد و اولین حرف نام او را چاپ کند.
		
		\item برنامه‌ای بنویسید که دو عدد از ورودی بگیرد و بررسی کند که آیا حاصل‌جمع آن‌ها عددی فرد است یا زوج.
		\[
		x = a + b, \quad x \, \text{زوج است اگر: } x \bmod 2 = 0
		\]
		
		\item برنامه‌ای بنویسید که سه عدد از ورودی بگیرد و بزرگترین عدد را نمایش دهد.
		\[
		\text{بزرگترین عدد} = \max(a, b, c)
		\]
		
		\item برنامه‌ای بنویسید که نام کاربر را از ورودی بگیرد و اگر نام او شامل حرف "ا" بود، پیغام "حرف ا در نام شما وجود دارد" را چاپ کند.
		
		\item برنامه‌ای بنویسید که یک عدد صحیح از ورودی بگیرد و اگر عدد فرد بود، آن را به عدد زوج بعدی تبدیل کند و نمایش دهد.
		\[
		x_{\text{جدید}} = x + 1 \quad \text{اگر } x \bmod 2 \neq 0
		\]
		
		\item برنامه‌ای بنویسید که دو عدد از ورودی بگیرد و بررسی کند که آیا هر دو عدد هم‌علامت (مثبت یا منفی) هستند یا خیر.
		
		\item برنامه‌ای بنویسید که عددی از ورودی بگیرد و بررسی کند که آیا آن عدد بر ۳ یا ۵ بخش‌پذیر است یا خیر.
		\[
		x \, \text{بر ۳ یا ۵ بخش‌پذیر است اگر: } x \bmod 3 = 0 \, \text{یا} \, x \bmod 5 = 0
		\]
		
		\item برنامه‌ای بنویسید که یک رشته از ورودی بگیرد و تعداد کاراکترهای آن رشته را نمایش دهد.
		
		\item برنامه‌ای بنویسید که سن کاربر را از ورودی بگیرد و اگر بین ۱۳ تا ۱۹ باشد، پیغام "شما نوجوان هستید" را چاپ کند.
		
		\item برنامه‌ای بنویسید که یک عدد اعشاری از ورودی بگیرد و آن را به نزدیک‌ترین عدد صحیح گرد کند و نمایش دهد.
		\[
		x_{\text{گرد شده}} = \text{round}(x)
		\]
		
		\item برنامه‌ای بنویسید که سه عدد از ورودی بگیرد و بررسی کند که آیا مجموع آن‌ها بین ۱۰۰ و ۲۰۰ است یا خیر.
		\[
		100 \leq (a + b + c) \leq 200
		\]
		
	\end{enumerate}
	
	\newpage
	\section*{تمرین‌های سری چهارم}
	\begin{enumerate}
		\item برنامه‌ای بنویسید که از ۱ تا ۱۰ را چاپ کند.
		
		\item برنامه‌ای بنویسید که از عددی که کاربر وارد می‌کند تا ۱ شمارش معکوس انجام دهد.
		
		\item برنامه‌ای بنویسید که مجموع اعداد از ۱ تا ۲۰ را محاسبه کرده و نمایش دهد.
		\[
		S = \sum_{i=1}^{20} i
		\]
		
		\item برنامه‌ای بنویسید که تمامی اعداد فرد از ۱ تا ۵۰ را چاپ کند.
		
		\item برنامه‌ای بنویسید که تعداد ارقام یک عدد را از کاربر بگیرد و تمام ارقام آن را چاپ کند.
		
		\item برنامه‌ای بنویسید که مجموع اعداد فرد از ۱ تا ۱۰۰ را محاسبه کرده و نمایش دهد.
		\[
		S = \sum_{i=1, \, i \, \text{فرد}}^{100} i
		\]
		
		\item برنامه‌ای بنویسید که اعداد از ۲ تا ۲۰ را که بر ۳ بخش‌پذیر هستند، چاپ کند.
		
		\item برنامه‌ای بنویسید که کاربر را تا زمانی که عددی منفی وارد کند، از او عدد بخواهد.
		
		\item برنامه‌ای بنویسید که اعداد از ۱ تا ۵۰۰ را که بر ۴ بخش‌پذیر هستند، چاپ کند.
		
		\item برنامه‌ای بنویسید که تعداد اعداد مثبت وارد شده توسط کاربر را تا زمانی که عدد ۰ وارد کند، محاسبه کند.
		
		\item برنامه‌ای بنویسید که تا ۵۰ عدد اول را پیدا کرده و چاپ کند.
		
		\item برنامه‌ای بنویسید که از کاربر سه عدد بگیرد و بزرگترین عدد را با استفاده از حلقه‌ها پیدا کند.
		
		\item برنامه‌ای بنویسید که تمامی اعداد زوج بین ۱ تا ۵۰ را با استفاده از حلقه چاپ کند.
		
		\item برنامه‌ای بنویسید که مجموع اعداد از ۱ تا ۵۰ را با استفاده از حلقه \texttt{while} محاسبه کند.
		\[
		S = \sum_{i=1}^{50} i
		\]
		
		\item برنامه‌ای بنویسید که تمام اعداد مضاعف ۷ بین ۱ تا ۱۰۰ را چاپ کند.
		
		\item برنامه‌ای بنویسید که تعداد اعداد منفی وارد شده توسط کاربر را تا زمانی که عدد مثبت وارد کند، محاسبه کند.
		
		\item برنامه‌ای بنویسید که اعداد ۱ تا ۱۰۰ را چاپ کند، اما برای اعداد قابل بخش‌پذیر بر ۳، کلمه "Fizz" و برای اعداد قابل بخش‌پذیر بر ۵، کلمه "Buzz" را چاپ کند.
		
		\item برنامه‌ای بنویسید که تعداد اعداد اول از ۱ تا ۱۰۰ را با استفاده از حلقه‌ها محاسبه کند.
		
		\item برنامه‌ای بنویسید که از کاربر بخواهد تا عددی را وارد کند و سپس برنامه تمام مضارب آن عدد تا ۱۰۰ را چاپ کند.
		\[
		x_i = n \cdot i \quad \text{که } x_i \leq 100
		\]
		
		\item برنامه‌ای بنویسید که از کاربر بخواهد یک عدد وارد کند و سپس تمام اعداد از ۱ تا آن عدد را که بر ۲ بخش‌پذیر نیستند را چاپ کند.

		\item برنامه ای بنویسید که تعداد کلمات تکرار شده در یک رشته را پیدا کند.
		
		\item برنامه‌ای بنویسید که با استفاده از کاراکتر * اشکال هندسی مختلف، از جمله مربع، مستطیل، مثلث متساوی‌الساقین (توپر و توخالی)، و متوازی‌الأضلاع را ترسیم کند.

		\item برنامه ای بنویسید که یک حاشیه دور ترمینال (صفحه خروجی) ایجاد کند. و در وسط آن به نمایش یک متن دلخواه بپردازد.
	
		\item برنامه‌ای بنویسید که یک متن متحرک را در اطراف ترمینال (صفحه خروجی) به نمایش بگذارد. این متن باید به‌آرامی حرکت کند و با تغییر رنگ‌های مختلف نمایش داده شود تا جلوه‌ای پویا و جذاب ایجاد کند.

		\item برنامه‌ای بنویسید که با استفاده از کاراکترهای مشخص، یک رودخانه را در صفحه ترمینال رسم کند. طراحی این برنامه باید به‌گونه‌ای باشد که تغییرات عرض رودخانه در هنگام اجرای برنامه به وضوح قابل مشاهده باشد و تغییرات در طول زمان به‌طور داینامیک نمایش داده شود.
		
	\end{enumerate}
	
	
	
	\newpage
	\section*{تمرین‌های سری پنجم}
	\begin{enumerate}
		
		
		\item برنامه‌ای بنویسید که اعداد فیبوناچی تا \(n\) را چاپ کند. 
		\[
		F(n) = F(n-1) + F(n-2) \quad \text{با شرایط اولیه: } F(0) = 0, \, F(1) = 1
		\]
		
		\item برنامه‌ای بنویسید که تعداد ارقام زوج یک عدد را محاسبه کند و چاپ کند.
		\[
		\text{تعداد ارقام زوج } = \sum_{i=1}^{n} \text{اگر } d_i \mod 2 = 0
		\]
		
		\item برنامه‌ای بنویسید که معکوس یک رشته را چاپ کند.
		
		\item برنامه‌ای بنویسید که اعداد اول بین دو عدد داده شده را چاپ کند.
		
		\item برنامه‌ای بنویسید که جمع اعداد فرد بین \(n\) و \(m\) را محاسبه کند و چاپ کند.
		\[
		S = \sum_{i=n}^{m} i \quad \text{برای } i \text{ که فرد است.}
		\]
		
		\item برنامه‌ای بنویسید که تعداد ارقام فرد یک عدد را محاسبه کند و چاپ کند.
		\[
		\text{تعداد ارقام فرد } = \sum_{i=1}^{n} \text{اگر } d_i \mod 2 \neq 0
		\]
		
		\item برنامه‌ای بنویسید که عدد اول بعد از \(n\) را پیدا کند و چاپ کند.
		
		\item برنامه‌ای بنویسید که معکوس لیستی از اعداد را چاپ کند.
		
		\item برنامه‌ای بنویسید که عوامل یک عدد را شناسایی کرده و چاپ کند.
		\[
		d \quad \text{که } n \mod d = 0
		\]
		
		\item برنامه‌ای بنویسید که عددی را از کاربر بگیرد و بزرگترین عامل آن را چاپ کند.
		\[
		\text{بزرگترین عامل } = \max(d \quad \text{که } n \mod d = 0)
		\]
		
		\item برنامه‌ای بنویسید که عدد اول بعد از یک عدد داده شده را چاپ کند.
		
		\item برنامه‌ای بنویسید که تعداد ارقام متفاوت یک عدد را محاسبه کرده و چاپ کند.
		
		\item برنامه‌ای بنویسید که اولین عدد بین \(n\) و \(m\) که بر \(k\) بخش‌پذیر است را چاپ کند.
		\[
		x \quad \text{که} x \mod k = 0
		\]
		
		\item برنامه‌ای بنویسید که اعداد اول بین \(m\) و \(n\) را محاسبه کرده و چاپ کند.
		
		\item برنامه‌ای بنویسید که حد بالایی عدد اول \(k\)-ام را محاسبه کرده و چاپ کند.
		
		\item برنامه‌ای بنویسید که معکوس یک عدد را بدون استفاده از رشته تبدیل کند و چاپ کند.
		
		\item برنامه‌ای بنویسید که مجموع ارقام یک عدد را محاسبه کند و چاپ کند.
		\[
		S = \sum_{i=1}^{n} d_i \quad
		\]
		
		
		\begin{center}
			که \( d_i \) ارقام عدد \( n \) هستند.
		\end{center}
		
		
		\item برنامه‌ای بنویسید که گرمایش بدن انسان را بر اساس دمای داده شده تبدیل به فارنهایت کند.
		\[
		F = \frac{9}{5}C + 32
		\]
		
		\item برنامه‌ای بنویسید که دنباله زمانی را با استفاده از حلقه \texttt{while} بررسی کند و وضعیت آن را چاپ کند.
		
		\item 	برنامه‌ای بنویسید که عدد \(n\) را از کاربر بگیرد و مجموع ارقام آن را محاسبه و چاپ کند.
		\[
		S = \sum_{i=1}^{n} d_i \quad \text{که } d_i
		\]
		
		\begin{center}
			که \( d_i \) ارقام عدد \( n \) هستند.
		\end{center}
		
		
	\end{enumerate}
	
	
	
	\newpage
	\section*{تمرین‌های سری ششم}
	
	\subsection*{تمرین‌های مربوط به \lr{List}}
	
	\begin{enumerate}
		\item لیستی از اعداد را تعریف کنید و مجموع تمام اعداد موجود در لیست را محاسبه کنید.
		\item یک لیست از اعداد را دریافت کنید و کوچکترین و بزرگترین عدد موجود در لیست را پیدا کنید.
		\item لیستی از اعداد را دریافت کنید و لیستی جدید بسازید که شامل اعداد زوج لیست اصلی باشد.
		\item یک لیست از اعداد را دریافت کنید و ترتیب آن را معکوس کنید.
		\item لیستی از رشته‌ها را دریافت کنید و طولانی‌ترین رشته موجود در لیست را پیدا کنید.
		\item لیستی از اعداد را دریافت کنید و اعداد تکراری را حذف کنید تا لیستی با اعداد یکتا به دست آید.
		\item لیستی از اعداد را دریافت کنید و مجموع اعداد در ایندکس‌های زوج را محاسبه کنید.
		\item دو لیست را دریافت کنید و آنها را به یک لیست جدید ادغام کنید.
		\item لیستی از رشته‌ها را دریافت کنید و آن را به ترتیب حروف الفبا مرتب کنید.
		\item یک لیست دو‌بعدی ایجاد کنید که مقادیر آن شامل حاصل ضرب اندیس‌های سطر و ستون باشد.
	\end{enumerate}
	
	\subsection*{تمرین‌های مربوط به \lr{Tuple}}
	
	\begin{enumerate}
		\item یک \lr{tuple} شامل اعداد ایجاد کنید و مجموع اعداد آن را محاسبه کنید.
		\item یک \lr{tuple} دریافت کنید و بررسی کنید آیا یک مقدار خاص در آن موجود است یا خیر.
		\item یک \lr{tuple} ایجاد کنید که شامل سه \lr{tuple} داخلی باشد و مقدار خاصی را از هر کدام استخراج کنید.
		\item یک \lr{tuple} شامل اعداد دریافت کنید و کوچکترین و بزرگترین مقدار را بیابید.
		\item یک \lr{tuple} شامل اعداد را به یک لیست تبدیل کنید و مقدار جدیدی به آن اضافه کنید.
		\item یک \lr{tuple} دو‌بعدی شامل مختصات ایجاد کنید و فاصله بین دو نقطه خاص را محاسبه کنید.
		\item از یک رشته، حروف آن را در یک \lr{tuple} قرار دهید.
		\item دو \lr{tuple} ایجاد کنید و آنها را به یک \lr{tuple} جدید ادغام کنید.
		\item بررسی کنید که آیا دو \lr{tuple} مشخص دقیقاً مشابه هستند یا خیر.
		\item طول یک \lr{tuple} را بدون استفاده از تابع \lr{len} محاسبه کنید.
	\end{enumerate}
	
	\subsection*{تمرین‌های مربوط به \lr{Set}}
	
	\begin{enumerate}
		\item یک \lr{set} شامل اعداد ایجاد کنید و مجموع مقادیر آن را محاسبه کنید.
		\item دو \lr{set} ایجاد کنید و اشتراک آنها را پیدا کنید.
		\item دو \lr{set} ایجاد کنید و اجتماع آنها را محاسبه کنید.
		\item دو \lr{set} ایجاد کنید و تفاوت آنها را بیابید.
		\item یک \lr{set} ایجاد کنید و یک مقدار جدید به آن اضافه کنید.
		\item یک \lr{set} ایجاد کنید و بررسی کنید که آیا یک مقدار خاص در آن وجود دارد یا خیر.
		\item دو \lr{set} ایجاد کنید و بررسی کنید آیا یکی زیرمجموعه دیگری است یا خیر.
		\item لیستی از اعداد تکراری دریافت کنید و آنها را به یک \lr{set} تبدیل کنید.
		\item یک \lr{set} شامل اعداد را به یک لیست مرتب تبدیل کنید.
		\item یک \lr{set} ایجاد کنید و کوچکترین مقدار آن را بیابید.
	\end{enumerate}
	
	\subsection*{تمرین‌های مربوط به \lr{Dictionary}}
	
	\begin{enumerate}
		\item یک دیکشنری ایجاد کنید که شامل اسامی افراد و سن آنها باشد و سن یک فرد خاص را چاپ کنید.
		\item یک دیکشنری شامل اسامی افراد و نمرات آنها ایجاد کنید و میانگین نمرات را محاسبه کنید.
		\item یک دیکشنری ایجاد کنید و کلید جدیدی به آن اضافه کنید.
		\item یک دیکشنری ایجاد کنید و بررسی کنید که آیا کلید خاصی در آن وجود دارد یا خیر.
		\item دیکشنری‌ای شامل کلمات و تعداد تکرار آنها از یک متن ایجاد کنید.
		\item یک دیکشنری شامل محصولات و قیمت آنها ایجاد کنید و گران‌ترین محصول را بیابید.
		\item یک دیکشنری ایجاد کنید و کلیدها و مقادیر آن را جداگانه چاپ کنید.
		\item دیکشنری‌ای شامل اسامی دانش‌آموزان و نمرات آنها ایجاد کنید و دانش‌آموزانی که نمره بالای ۱۵ دارند را بیابید.
		\item دیکشنری‌ای شامل نام و شماره تماس افراد ایجاد کنید و شماره تماس یک فرد خاص را پیدا کنید.
		\item دو دیکشنری را ادغام کنید و دیکشنری نهایی را چاپ کنید.
	\end{enumerate}
	
	
	
	\newpage
	\section*{تمرین‌های سری هفتم}
	
	\subsection*{تمرین‌های مربوط به \lr{List}}
	
	\begin{enumerate}
		\item لیستی از اعداد تولید کنید که هر عضو آن، مجموع اعضای قبلی لیست باشد (اعداد فیبوناچی ساده).
		\item لیستی از اعداد شامل اعداد تکراری بسازید و تعداد تکرار هر عدد را در یک لیست جدید ذخیره کنید.
		\item یک لیست دو‌بعدی از اعداد بسازید و جمع هر سطر و ستون را محاسبه کنید.
		\item یک لیست از رشته‌ها بسازید و لیست جدیدی ایجاد کنید که هر رشته به صورت معکوس باشد.
		\item لیستی از اعداد بسازید و اعدادی که در شاخص‌های فرد هستند را به صورت مجزا مرتب کنید.
		\item یک لیست شامل مقادیر متنی و عددی بسازید و فقط مقادیر عددی آن را جمع بزنید.
		\item دو لیست بسازید و مقادیر مشترک آنها را در یک لیست جدید ذخیره کنید.
		\item یک لیست شامل نام دانش‌آموزان و نمرات آنها بسازید و دانش‌آموزانی که نمره بالای میانگین دارند را پیدا کنید.
		\item یک لیست دو‌بعدی بسازید که مقادیر هر سلول، میانگین سطر و ستون آن باشد.
		\item لیستی از مقادیر اعداد تصادفی بسازید و به ترتیب صعودی، نزولی و به صورت مرتب‌سازی تصادفی لیست را چاپ کنید.
	\end{enumerate}
	
	\subsection*{تمرین‌های مربوط به \lr{Tuple}}
	
	\begin{enumerate}
		\item یک \lr{tuple} شامل اعداد ایجاد کنید و تعداد اعداد زوج و فرد آن را جداگانه محاسبه کنید.
		\item یک \lr{tuple} دریافت کنید و معکوس آن را بدون تبدیل به لیست چاپ کنید.
		\item یک \lr{tuple} شامل چند \lr{tuple} داخلی بسازید و مجموع مقادیر در هر \lr{tuple} داخلی را به صورت جداگانه محاسبه کنید.
		\item یک \lr{tuple} ایجاد کنید و اولین و آخرین عنصر آن را جابجا کنید.
		\item دو \lr{tuple} شامل اعداد بسازید و مجموع عضوهای متناظر آنها را در یک \lr{tuple} جدید ذخیره کنید.
		\item یک \lr{tuple} شامل رشته‌ها بسازید و طول هر رشته را در یک \lr{tuple} دیگر ذخیره کنید.
		\item یک \lr{tuple} شامل مختصات چند نقطه در فضا بسازید و فاصله میانگین نقاط را از مبدا پیدا کنید.
		\item یک \lr{tuple} ایجاد کنید و هر عضو آن را به صورت معکوس در یک \lr{tuple} جدید ذخیره کنید.
		\item یک \lr{tuple} از اعداد بسازید و تمام مقادیر بین دو عدد مشخص را از آن حذف کنید.
		\item یک \lr{tuple} شامل مقادیر تکراری بسازید و فقط مقادیر یکتا را به صورت یک \lr{tuple} جدید ذخیره کنید.
	\end{enumerate}
	
	\subsection*{تمرین‌های مربوط به \lr{Set}}
	
	\begin{enumerate}
		\item یک \lr{set} شامل اعداد ایجاد کنید و مجموع اعداد زوج و فرد را به صورت جداگانه محاسبه کنید.
		\item دو \lr{set} شامل اعداد ایجاد کنید و بررسی کنید که آیا یکی زیرمجموعه دیگری است یا خیر.
		\item یک \lr{set} ایجاد کنید و تمام مقادیر آن را به توان دو برسانید.
		\item یک \lr{set} شامل اعداد تصادفی بسازید و فقط مقادیر اول آن را در یک \lr{set} جدید ذخیره کنید.
		\item دو \lr{set} بسازید و اعدادی که در هیچ کدام وجود ندارند را پیدا کنید.
		\item یک \lr{set} ایجاد کنید و مجموع مقادیر کوچک‌تر از یک عدد مشخص را پیدا کنید.
		\item یک \lr{set} شامل رشته‌ها بسازید و تمام رشته‌های شروع‌شده با یک حرف خاص را در یک \lr{set} جدید ذخیره کنید.
		\item یک \lr{set} ایجاد کنید و هر عضو آن را در یک لیست مرتب ذخیره کنید.
		\item دو \lr{set} شامل اعداد بسازید و اختلاف متقارن آنها را محاسبه کنید.
		\item یک \lr{set} از رشته‌ها بسازید و رشته‌ای که بیشترین تعداد کاراکتر دارد را پیدا کنید.
	\end{enumerate}
	
	\subsection*{تمرین‌های مربوط به \lr{Dictionary}}
	
	\begin{enumerate}
		\item یک دیکشنری شامل اسامی دانش‌آموزان و نمرات آنها بسازید و نام دانش‌آموزی با بالاترین نمره را پیدا کنید.
		\item یک دیکشنری شامل محصولات و قیمت آنها بسازید و میانگین قیمت‌ها را محاسبه کنید.
		\item یک دیکشنری شامل رشته‌ها و تعداد تکرار آنها از یک متن بسازید و پرکاربردترین کلمه را پیدا کنید.
		\item دیکشنری‌ای شامل نام و شماره تماس افراد بسازید و شماره تماس افراد را به ترتیب حروف الفبا چاپ کنید.
		\item یک دیکشنری شامل اطلاعات محصولات بسازید و محصولاتی که قیمت آنها بالاتر از مقدار مشخصی است را پیدا کنید.
		\item یک دیکشنری شامل کلمات و تعداد تکرار آنها بسازید و دیکشنری‌ای جدید ایجاد کنید که فقط شامل کلمات با تعداد تکرار بالای ۵ باشد.
		\item یک دیکشنری ایجاد کنید و کلیدها را به ترتیب حروف الفبا مرتب کنید.
		\item یک دیکشنری شامل اعداد و مقادیرشان بسازید و فقط مقادیر زوج را در دیکشنری‌ای جدید ذخیره کنید.
		\item دو دیکشنری شامل اطلاعات مشابه بسازید و مواردی که در هر دو مشترک هستند را پیدا کنید.
		\item دیکشنری‌ای شامل اطلاعات چند نفر بسازید و اطلاعات شخصی که طولانی‌ترین نام را دارد پیدا کنید.
	\end{enumerate}
	
	
	
	\newpage
	\section*{تمرین‌های سری هشتم}
	
	\subsection*{تمرین‌های مربوط به \lr{List}}
	
	\begin{enumerate}
		\item یک لیست از اعداد بسازید و بزرگترین و کوچکترین عدد را پیدا کرده، آنها را جابجا کنید.
		\item لیستی از رشته‌ها بسازید و تمام رشته‌هایی که بیش از ۵ حرف دارند را در لیست جدیدی ذخیره کنید.
		\item یک لیست دو‌بعدی ایجاد کنید و مجموع هر سطر و ستون را در یک لیست جدید ذخیره کنید.
		\item یک لیست از اعداد بسازید که از اعداد اول تا ۱۰۰ پر شود و تنها اعداد اول آن را جدا کنید.
		\item لیستی از اعداد تصادفی بسازید و لیست جدیدی بسازید که فقط شامل اعداد فرد باشد.
		\item لیستی از رشته‌ها بسازید و هر رشته را به حروف کوچک تبدیل کنید.
		\item دو لیست بسازید و لیستی جدید بسازید که مقادیر مشترک و غیرمشترک این دو لیست را در خود جای دهد.
		\item یک لیست از اعداد بسازید که مقدار هر عضو آن دو برابر مقدار قبلی آن باشد.
		\item لیستی از اعداد بسازید و همه اعداد تکراری آن را حذف کنید.
		\item لیستی از رشته‌ها بسازید و هر رشته‌ای که شروع به حرف "A" یا "a" دارد را در لیست جدید ذخیره کنید.
	\end{enumerate}
	
	\subsection*{تمرین‌های مربوط به \lr{Tuple}}
	
	\begin{enumerate}
		\item یک \lr{tuple} از اعداد بسازید و جمع اعضای آن را محاسبه کنید.
		\item یک \lr{tuple} از ۳ \lr{tuple} داخلی بسازید و مقادیر خاصی را از هرکدام استخراج کنید.
		\item یک \lr{tuple} شامل اعداد ایجاد کنید و تعداد اعداد بزرگتر از مقدار مشخص را محاسبه کنید.
		\item یک \lr{tuple} شامل چندین رشته بسازید و طول هر رشته را در یک \lr{tuple} جدید ذخیره کنید.
		\item دو \lr{tuple} از رشته‌ها بسازید و ترتیب آنها را معکوس کنید.
		\item یک \lr{tuple} ایجاد کنید که هر عضو آن از یک لیست به دست آمده باشد و ترتیب آن را تغییر دهید.
		\item یک \lr{tuple} از مختصات نقطه‌ها بسازید و فاصله بین اولین و آخرین نقطه را محاسبه کنید.
		\item یک \lr{tuple} شامل اعداد ایجاد کنید و فقط مقادیر بزرگتر از میانگین آن را ذخیره کنید.
		\item یک \lr{tuple} شامل مقادیر تصادفی بسازید و فقط مقادیر مثبت آن را ذخیره کنید.
		\item یک \lr{tuple} از مقادیر رشته‌ای بسازید و تمام رشته‌هایی که بیشتر از ۴ حرف دارند را پیدا کنید.
	\end{enumerate}
	
	\subsection*{تمرین‌های مربوط به \lr{Set}}
	
	\begin{enumerate}
		\item دو \lr{set} از اعداد بسازید و تفاوت آنها را در یک \lr{set} جدید ذخیره کنید.
		\item یک \lr{set} ایجاد کنید و تمام مقادیر آن را به صورت تصادفی در یک لیست مرتب کنید.
		\item یک \lr{set} از رشته‌ها بسازید و رشته‌هایی که شروع به حرف خاصی دارند را پیدا کنید.
		\item دو \lr{set} از اعداد بسازید و تنها اعدادی که در هر دو وجود دارند را در یک \lr{set} جدید ذخیره کنید.
		\item یک \lr{set} ایجاد کنید و اعداد آن را به توان دوم برسانید.
		\item یک \lr{set} از مقادیر عددی بسازید و همه مقادیر کمتر از یک عدد خاص را حذف کنید.
		\item یک \lr{set} از مقادیر رشته‌ای بسازید و تنها مقادیر با طول بیش از ۴ حرف را در یک \lr{set} جدید ذخیره کنید.
		\item یک \lr{set} از اعداد بسازید و مجموع اعداد زوج و فرد را به صورت جداگانه محاسبه کنید.
		\item یک \lr{set} از اعداد اول بسازید و مقادیر آن را به ترتیب صعودی مرتب کنید.
		\item یک \lr{set} ایجاد کنید و مقادیر آن را به لیست تبدیل کرده و به ترتیب حروف الفبا مرتب کنید.
	\end{enumerate}
	
	\subsection*{تمرین‌های مربوط به \lr{Dictionary}}
	
	\begin{enumerate}
		\item یک دیکشنری از نام‌ها و نمرات دانش‌آموزان بسازید و میانگین نمرات را محاسبه کنید.
		\item یک دیکشنری شامل محصولات و قیمت آنها بسازید و محصولاتی که قیمت آنها از میانگین بیشتر است را بیابید.
		\item یک دیکشنری ایجاد کنید و کلیدهایی را به آن اضافه کنید که حروف اول آنها "B" یا "b" باشد.
		\item دیکشنری‌ای شامل کلمات و تعداد تکرار آنها بسازید و تنها کلمات با تعداد تکرار بیش از ۵ را پیدا کنید.
		\item یک دیکشنری شامل نام‌ها و سن‌ها بسازید و افراد بالای ۳۰ سال را در یک دیکشنری جدید ذخیره کنید.
		\item دو دیکشنری از اطلاعات مشابه بسازید و تفاوت آنها را پیدا کنید.
		\item یک دیکشنری شامل محصولات و تعداد موجودی هرکدام بسازید و محصولاتی که موجودی آن‌ها صفر است را پیدا کنید.
		\item یک دیکشنری شامل نام افراد و شماره‌های تماس آنها بسازید و شماره تماس فردی با نام خاص را پیدا کنید.
		\item یک دیکشنری ایجاد کنید و تمامی کلیدها را به ترتیب الفبایی چاپ کنید.
		\item یک دیکشنری شامل نام کتاب‌ها و قیمت آنها بسازید و کتاب‌هایی با قیمت بیشتر از ۵۰ هزار تومان را پیدا کنید.
	\end{enumerate}
	
	
	
	\newpage
	\section*{تمرین‌های سری نهم: کار با فایل‌ها}
	
	\subsection*{سوالات ساده‌تر:}
	
	\begin{enumerate}
		\item یک فایل متنی جدید با نام \lr{test.txt} بسازید و یک جمله ساده مانند "سلام دنیا!" در آن بنویسید.
		\item یک فایل متنی بسازید و نام ۵ نفر را در آن بنویسید. سپس فایل را ذخیره کرده و محتویات آن را چاپ کنید.
		\item یک فایل متنی ایجاد کنید و در آن ۱۰ عدد تصادفی بنویسید. سپس محتویات آن را با استفاده از یک برنامه نمایش دهید.
		\item فایلی بسازید که شامل اطلاعات یک محصول باشد (نام، قیمت، توضیحات). سپس آن را به صورت خط به خط بخوانید و چاپ کنید.
		\item یک فایل متنی بسازید و محتوای آن را با استفاده از کد برنامه درون یک متغیر ذخیره کنید.
		\item یک فایل متنی ایجاد کنید و آن را به صورت خط به خط باز کنید و هر خط را چاپ کنید.
		\item یک فایل متنی بسازید و متن ساده‌ای در آن بنویسید. سپس این فایل را باز کرده و تعداد کلمات موجود در آن را محاسبه کنید.
		\item فایلی بسازید که شامل نام و سن ۵ نفر باشد. سپس آن را به صورت خط به خط بخوانید و هر خط را چاپ کنید.
		\item یک فایل متنی ایجاد کنید و چند خط ساده در آن بنویسید. سپس تعداد خطوط موجود در فایل را محاسبه کنید.
		\item یک فایل متنی بسازید و رشته‌ای خاص را در آن جستجو کنید.
	\end{enumerate}
	
	\subsection*{سوالات متوسط:}
	
	\begin{enumerate}
		\item یک فایل متنی بسازید و نام، سن و شهر افراد را در آن ذخیره کنید. سپس آن را به صورت خط به خط بخوانید و برای هر فرد، یک جمله چاپ کنید.
		\item یک فایل متنی بسازید که شامل ۵ جمله باشد. سپس یک کلمه خاص را در تمام فایل جستجو کنید و تعداد دفعات وجود آن را محاسبه کنید.
		\item یک فایل متنی بسازید و محتوای آن را به طور معکوس چاپ کنید.
		\item یک فایل متنی ایجاد کنید و آن را با استفاده از برنامه باز کرده و محتویات آن را به طور خط به خط در یک لیست ذخیره کنید.
		\item یک فایل متنی ایجاد کنید و هر خط آن را با یک عدد تصادفی به انتهای آن اضافه کنید.
		\item فایلی بسازید که شامل اسامی ۵ نفر باشد. سپس اسامی را به ترتیب حروف الفبا مرتب کرده و در فایل ذخیره کنید.
		\item یک فایل متنی بسازید که شامل یک سری اعداد باشد. سپس از برنامه استفاده کنید تا بزرگترین و کوچکترین عدد را از فایل استخراج کنید.
		\item یک فایل متنی ایجاد کنید و برای هر خط آن، ابتدا تعداد کاراکترها و سپس تعداد کلمات موجود در آن را چاپ کنید.
		\item یک فایل متنی بسازید و محتوای آن را با استفاده از برنامه به یک آرایه تبدیل کرده و آن را چاپ کنید.
		\item فایلی بسازید و پس از نوشتن اطلاعات در آن، فایل را باز کرده و محتویات آن را به ترتیب چاپ کنید.
	\end{enumerate}
	
	\subsection*{سوالات پیچیده‌تر:}
	
	\begin{enumerate}
		\item یک فایل متنی بسازید که شامل لیستی از دانش‌آموزان و نمرات آنها باشد. سپس برنامه‌ای بنویسید که نام دانش‌آموزانی که نمرات بالاتر از میانگین دارند را چاپ کند.
		\item یک فایل متنی بسازید که شامل جمله‌ای خاص باشد. سپس در انتهای آن جمله، یک جمله جدید اضافه کنید.
		\item یک فایل متنی ایجاد کنید که شامل اسامی کتاب‌ها و نویسندگان آنها باشد. سپس یک کتاب جدید به انتهای فایل اضافه کنید.
		\item یک فایل متنی ایجاد کنید که حاوی نام محصولات و قیمت آنها باشد. سپس قیمت تمام محصولات را به ۱۰ درصد افزایش دهید و فایل را به روز کنید.
		\item یک فایل متنی بسازید که شامل یک جمله باشد. سپس تعداد تکرار هر کلمه در جمله را محاسبه کرده و نتیجه را در فایل ذخیره کنید.
		\item یک فایل متنی ایجاد کنید که حاوی تاریخ‌های مختلف باشد. سپس برای هر تاریخ، روز هفته آن را محاسبه کرده و در انتهای آن اضافه کنید.
		\item یک فایل متنی بسازید که شامل یک لیست از اعداد باشد. سپس آنها را به صورت معکوس در فایل بنویسید.
		\item یک فایل متنی ایجاد کنید و آن را به گونه‌ای ویرایش کنید که هر عدد زوج را به "عدد زوج" و هر عدد فرد را به "عدد فرد" تغییر دهید.
		\item یک فایل متنی ایجاد کنید و برای هر خط، تاریخ و زمان فعلی را به آن اضافه کنید.
		\item یک فایل متنی بسازید که شامل نام کتاب‌ها و تعداد صفحات آنها باشد. سپس نام کتاب‌هایی که بیش از ۳۰۰ صفحه دارند را به انتهای فایل اضافه کنید.
	\end{enumerate}
	
	\subsection*{سوالات نهایی:}
	
	\begin{enumerate}
		\item یک فایل متنی بسازید که شامل لیستی از ایمیل‌ها باشد. سپس برنامه‌ای بنویسید که ایمیل‌هایی که از دامنه خاصی هستند را فیلتر کرده و چاپ کند.
		\item یک فایل متنی ایجاد کنید که شامل اسامی شهرها باشد. سپس اسامی شهرهایی که در آنها حرف "A" وجود دارد را در فایل جداگانه ذخیره کنید.
		\item یک فایل متنی ایجاد کنید که شامل نام کشورها و پایتخت‌های آنها باشد. سپس نام پایتخت‌ها را به ترتیب حروف الفبا مرتب کرده و در فایل ذخیره کنید.
		\item یک فایل متنی بسازید که شامل نام روزهای هفته باشد. سپس از طریق برنامه، روزهایی که بیشتر از ۵ حرف دارند را جدا کنید و در فایل جدید ذخیره کنید.
		\item یک فایل متنی ایجاد کنید که شامل نام مشتریان و تاریخ خریدهای آنها باشد. سپس مشتریانی که در یک تاریخ خاص خرید کرده‌اند را چاپ کنید.
		\item یک فایل متنی بسازید که شامل لیستی از محصولات و قیمت آنها باشد. سپس ارزان‌ترین و گران‌ترین محصول را از فایل استخراج کرده و چاپ کنید.
		\item یک فایل متنی ایجاد کنید که شامل چند رشته باشد. سپس از برنامه استفاده کنید تا همه رشته‌هایی که بیش از ۱۰ حرف دارند را چاپ کنید.
		\item یک فایل متنی بسازید که شامل تاریخ‌های مختلف باشد. سپس از طریق برنامه، تاریخ‌هایی که ماه آن‌ها "مهر" است را چاپ کنید.
		\item یک فایل متنی ایجاد کنید و در آن یک آدرس ایمیل بنویسید. سپس آن را ویرایش کرده و به آن نام کاربری و پسورد اضافه کنید.
		\item یک فایل متنی بسازید که شامل اطلاعات یک پروژه باشد (نام پروژه، تاریخ شروع، تاریخ پایان). سپس برای هر پروژه، مدت زمان انجام آن را محاسبه کنید.
		\item برنامه ای بنویسید که به تغییرات یک فایل از نظر محتوا بپردازد. و در صورت تغییر محتوا آن را چاپ کند.
	\end{enumerate}
	

	\newpage
	\section*{تمرین‌های سری دهم: کار با اکسل، JSON و اینترنت}
	
	\subsection*{کار با اکسل (Excel):}
	
	\begin{enumerate}
		\item یک فایل اکسل جدید بسازید و در آن اطلاعاتی مانند نام، سن و شغل ۵ نفر را وارد کنید. سپس فایل را ذخیره کنید.
		\item یک فایل اکسل ایجاد کنید و در آن لیستی از اعداد را وارد کنید. سپس میانگین این اعداد را با استفاده از فرمول اکسل محاسبه کنید.
		\item یک فایل اکسل ایجاد کنید و لیستی از دانش‌آموزان و نمرات آنها را وارد کنید. سپس به کمک فرمول اکسل، نمره‌ی متوسط کلاس را محاسبه کنید.
		\item یک فایل اکسل بسازید و در آن تاریخ تولد ۵ نفر را وارد کنید. سپس تعداد افراد بزرگتر از ۱۸ سال را محاسبه کنید.
		\item یک فایل اکسل ایجاد کنید که شامل لیستی از محصولات و قیمت‌های آنها باشد. سپس یک ستون برای محاسبه‌ی قیمت پس از اعمال تخفیف ۲۰ درصدی اضافه کنید.
		\item یک فایل اکسل بسازید که شامل نام روزهای هفته باشد. سپس روزهایی که حروف "A" دارند را در یک ستون جدید علامت‌گذاری کنید.
		\item یک فایل اکسل ایجاد کنید که شامل اسامی کتاب‌ها و تعداد صفحات آنها باشد. سپس کتاب‌هایی که بیش از ۳۰۰ صفحه دارند را جدا کنید.
		\item یک فایل اکسل بسازید که در آن تاریخ‌های مختلف وارد شده باشد. سپس تعداد تاریخ‌هایی که در ماه دی هستند را محاسبه کنید.
		\item یک فایل اکسل بسازید که شامل یک سری اعداد باشد. سپس به کمک فرمول اکسل، بزرگترین و کوچکترین عدد را از این سری پیدا کنید.
		\item یک فایل اکسل بسازید که شامل لیستی از فروش‌ها باشد. سپس مجموع فروش‌ها را محاسبه کنید و در سلول دیگری نمایش دهید.
	\end{enumerate}
	
	\subsection*{کار با JSON:}
	
	\begin{enumerate}
		\item یک فایل JSON بسازید که شامل اطلاعات یک شخص (نام، سن، شهر) باشد. سپس این فایل را با برنامه باز کرده و محتوای آن را چاپ کنید.
		\item یک فایل JSON ایجاد کنید که شامل یک لیست از محصولات و قیمت‌های آنها باشد. سپس قیمت هر محصول را ۱۰ درصد افزایش دهید و آن را ذخیره کنید.
		\item یک فایل JSON بسازید که شامل اطلاعات چند دانش‌آموز (نام، نمره) باشد. سپس دانش‌آموزانی که نمره بالای ۱۵ دارند را استخراج کنید.
		\item یک فایل JSON ایجاد کنید که شامل لیستی از کارمندان (نام، سمت، حقوق) باشد. سپس حقوق تمام کارمندان را به ۱۰ درصد افزایش دهید.
		\item یک فایل JSON بسازید که شامل تاریخ تولد افراد باشد. سپس افرادی که تاریخ تولدشان در ماه تیر است را استخراج کنید.
		\item یک فایل JSON بسازید که شامل لیستی از کشورها و پایتخت‌های آنها باشد. سپس نام کشورهایی که پایتخت آنها بیشتر از ۱۰ حرف دارد را استخراج کنید.
		\item یک فایل JSON بسازید که شامل یک سری اعداد باشد. سپس از برنامه استفاده کنید تا میانگین این اعداد را محاسبه کرده و آن را به فایل JSON اضافه کنید.
		\item یک فایل JSON بسازید که شامل لیستی از شهرها و جمعیت آنها باشد. سپس شهرهایی که جمعیتشان بیشتر از ۵۰۰۰۰۰ است را استخراج کنید.
		\item یک فایل JSON ایجاد کنید که شامل اطلاعات یک پروژه (نام پروژه، تاریخ شروع، تاریخ پایان) باشد. سپس مدت زمان پروژه را محاسبه کرده و به فایل اضافه کنید.
		\item یک فایل JSON بسازید که شامل یک سری رشته‌ها باشد. سپس طول هر رشته را محاسبه کرده و آن را به فایل JSON اضافه کنید.
	\end{enumerate}
	
	
	\subsection*{کار با اینترنت و درخواست‌های HTTP با پایتون (با استفاده از کتابخانه Requests):}
	
	\begin{enumerate}
		\item با استفاده از کتابخانه \lr{requests} پایتون، یک درخواست GET ارسال کرده و محتویات آن را چاپ کنید.
		\begin{flushleft}
			\url{https://jsonplaceholder.typicode.com/posts}
		\end{flushleft}
		\item با استفاده از کتابخانه \lr{requests} پایتون، یک درخواست GET ارسال کرده و اطلاعات مربوط به گیت‌هاب را چاپ کنید.
		\begin{flushleft}
			\url{https://api.github.com}
		\end{flushleft}
		\item با استفاده از کتابخانه \lr{requests} پایتون، درخواست GET ارسال کرده و نام و ایمیل کاربران را چاپ کنید.
		\begin{flushleft}
			\url{https://jsonplaceholder.typicode.com/users}
		\end{flushleft}
		\item با استفاده از کتابخانه \lr{requests} پایتون، یک درخواست GET ارسال کنید و عکس یک سگ تصادفی را دریافت و نمایش دهید.
		\begin{flushleft}
			\url{https://dog.ceo/api/breeds/image/random}
		\end{flushleft}
		\item با استفاده از کتابخانه \lr{requests} پایتون، درخواست GET ارسال کرده و وضعیت آب و هوا را چاپ کنید.
		\begin{flushleft}
			\url{https://www.metaweather.com/api/location/44418/}
		\end{flushleft}
		\item با استفاده از کتابخانه \lr{requests} پایتون، یک درخواست GET ارسال کرده و تنها مواردی که کامل نشده‌اند را چاپ کنید.
		\begin{flushleft}
			\url{https://jsonplaceholder.typicode.com/todos}
		\end{flushleft}
		\item با استفاده از کتابخانه \lr{requests} پایتون، یک درخواست GET ارسال کرده و قیمت بیت‌کوین را چاپ کنید.
		\begin{flushleft}
			\url{https://api.coindesk.com/v1/bpi/currentprice/BTC.json}
		\end{flushleft}
		\item با استفاده از کتابخانه \lr{requests} پایتون، درخواست GET ارسال کرده و اطلاعات آخرین پرتاب فضایی را دریافت کنید.
		\begin{flushleft}
			\url{https://api.spacexdata.com/v4/launches}
		\end{flushleft}
		\item با استفاده از کتابخانه \lr{requests} پایتون، یک درخواست GET ارسال کرده و لیست نژادهای سگ‌ها را چاپ کنید.
		\begin{flushleft}
			\url{https://dog.ceo/api/breeds/list/all}
		\end{flushleft}
		\item با استفاده از کتابخانه \lr{requests} پایتون، یک درخواست GET ارسال کرده و تنها آلبوم‌های مربوط به کاربر شماره 1 را چاپ کنید.
		\begin{flushleft}
			\url{https://jsonplaceholder.typicode.com/albums}
		\end{flushleft}
	\end{enumerate}
	
	
	
	
	\newpage
	\section*{تمرین‌های سری یازدهم: آشنایی با دیتابیس}
	
	\begin{enumerate}
		\item با استفاده از کتابخانه \lr{sqlite3} پایتون، به یک دیتابیس جدید به نام \texttt{example.db} متصل شوید.
		\item با استفاده از کتابخانه \lr{sqlite3}، یک جدول به نام \texttt{users} با ستون‌های \texttt{id}، \texttt{name}، و \texttt{age} ایجاد کنید.
		\item با استفاده از کتابخانه \lr{sqlite3}، یک رکورد جدید با مقادیر دلخواه در جدول \texttt{users} درج کنید.
		\item با استفاده از کتابخانه \lr{sqlite3}، تمام رکوردهای جدول \texttt{users} را بازیابی و چاپ کنید.
		\item با استفاده از کتابخانه \lr{sqlite3}، سن یک کاربر خاص را در جدول \texttt{users} تغییر دهید.
		\item با استفاده از کتابخانه \lr{sqlite3}، یک کاربر خاص را با استفاده از شناسه \texttt{id} حذف کنید.
		\item با استفاده از کتابخانه \lr{sqlite3}، تنها کاربران بالای ۱۸ سال را از جدول \texttt{users} دریافت و چاپ کنید.
		\item با استفاده از کتابخانه \lr{sqlite3}، بررسی کنید که آیا یک کاربر خاص در جدول \texttt{users} وجود دارد یا خیر.
		\item با استفاده از کتابخانه \lr{sqlite3}، تعداد کل کاربران موجود در جدول \texttt{users} را شمارش کرده و چاپ کنید.
		\item با استفاده از کتابخانه \lr{sqlite3}، جدول \texttt{users} را حذف کنید.
		\item برنامه‌ای بنویسید که نام کاربری و گذرواژه کاربران را دریافت کرده و آن‌ها را به طور امن در یک پایگاه داده sqlite3 ذخیره کند. گذرواژه باید به گونه‌ای رمزنگاری شود که با استفاده از الگوریتم bcrypt (نسخه 5 یا بالاتر) رمزنگاری شده و در آینده برای تطبیق گذرواژه هنگام ورود به سیستم مورد استفاده قرار گیرد.
	\end{enumerate}
	
	\newpage
	\section*{تمرین‌های سری دوازدهم: آشنایی با\lr{OpenCV}}
	
	\begin{enumerate}
		\item با استفاده از \lr{OpenCV}، یک تصویر را بارگذاری کرده و آن را به صورت سیاه و سفید تبدیل کنید.
		\item با استفاده از \lr{OpenCV}، یک ویدیو از وب‌کم دریافت کنید و آن را به صورت زنده نمایش دهید.
		\item با استفاده از \lr{OpenCV}، لبه‌های یک تصویر را با استفاده از الگوریتم \lr{Canny} تشخیص دهید.
		\item با استفاده از \lr{OpenCV}، چهره‌های موجود در یک تصویر را تشخیص داده و دور آنها مستطیل بکشید. از فایل \texttt{haarcascade\_frontalface\_default.xml} استفاده کنید.  
		لینک فایل:  
		\begin{flushleft}
			\url{https://github.com/opencv/opencv/blob/master/data/haarcascades/haarcascade_frontalface_default.xml}
		\end{flushleft}
		\item با استفاده از \lr{OpenCV}، یک پروژه ساده بنویسید که وسایل نقلیه را در یک تصویر شناسایی کند. از فایل \texttt{haarcascade\_car.xml} استفاده کنید.  
		لینک فایل:  
		\begin{flushleft}
			\url{https://github.com/opencv/opencv/blob/master/data/haarcascades/haarcascade_car.xml}
		\end{flushleft}
		\item با استفاده از \lr{OpenCV}، یک ویدیو دریافت کنید و تعداد اشیای متحرک موجود در هر فریم را شمارش کنید.
		\item با استفاده از \lr{OpenCV}، یک تصویر را به چندین بخش تقسیم کرده و میانگین رنگ هر بخش را محاسبه کنید.
		\item با استفاده از \lr{OpenCV}، یک تصویر را بارگذاری کرده و اشیای موجود در آن را با استفاده از الگوریتم \lr{Contour Detection} شناسایی کنید.
		\item با استفاده از \lr{OpenCV}، یک تصویر را بارگذاری کرده و نقاط گوشه در تصویر را با استفاده از الگوریتم \lr{Harris Corner Detection} پیدا کنید.
		\item با استفاده از \lr{OpenCV}، یک پروژه بنویسید که در یک ویدیو چهره‌ها را تشخیص داده و تعداد کل چهره‌های موجود در ویدیو را محاسبه کند.  
		لینک فایل:  
		\begin{flushleft}
			\url{https://github.com/opencv/opencv/blob/master/data/haarcascades/haarcascade_frontalface_default.xml}
		\end{flushleft}
	\end{enumerate}
	
	
	
	\newpage
		\section*{پروژه‌ها}
		
		در اینجا ده پروژه مختلف آورده شده است که دانشجویان می‌توانند روی آنها کار کنند. هر پروژه شامل توضیحات دقیق، ویژگی‌ها و مثال‌هایی است که به شما در درک بهتر پروژه کمک خواهد کرد.
		
		\begin{enumerate}
			
			\item \textbf{سیستم مدیریت کتابخانه}
			\begin{itemize}
				\item هدف: طراحی یک سیستم مدیریت کتابخانه که به کاربران اجازه می‌دهد کتاب‌ها را جستجو کرده، امانت بگیرند و بازگردانند.
				\item ویژگی‌ها:
				\begin{itemize}
					\item ثبت کتاب‌ها با ویژگی‌هایی مثل عنوان، نویسنده و دسته‌بندی.
					\item جستجو و فیلتر کتاب‌ها بر اساس عنوان یا نویسنده.
					\item قابلیت ثبت و پیگیری امانت کتاب‌ها.
					\item ارسال پیام‌های اطلاع‌رسانی به کاربران برای بازگرداندن کتاب‌ها.
				\end{itemize}
				\item مثال:
				\begin{itemize}
					\item کاربر می‌تواند با وارد کردن نام یک کتاب، اطلاعات آن را مشاهده کند و اگر در دسترس باشد، کتاب را امانت بگیرد.
					\item سیستم باید لیست کتاب‌های امانت گرفته را نمایش دهد و زمان بازگشت آنها را پیگیری کند.
				\end{itemize}
			\end{itemize}
			
			\item \textbf{مدیریت اطلاعات دانشجویی}
			\begin{itemize}
				\item هدف: طراحی سیستمی برای مدیریت اطلاعات دانشجویی شامل ثبت‌نام، نمرات و گزارش‌گیری.
				\item ویژگی‌ها:
				\begin{itemize}
					\item ثبت اطلاعات شخصی دانشجویان شامل نام، شماره دانشجویی، رشته و تاریخ‌تولد.
					\item ثبت نمرات و مشاهده وضعیت تحصیلی هر دانشجو.
					\item گزارش‌گیری از نمرات و مقایسه دانشجویان.
				\end{itemize}
				\item مثال:
				\begin{itemize}
					\item مدیر سیستم می‌تواند نمرات دانشجویان را وارد کرده و گزارش‌های تحصیلی آنها را دریافت کند.
					\item کاربران می‌توانند از طریق شماره دانشجویی یا نام، اطلاعات فردی هر دانشجو را مشاهده کنند.
				\end{itemize}
			\end{itemize}
			
			\item \textbf{سیستم رزرو بلیط سینما}
			\begin{itemize}
				\item هدف: طراحی سیستمی برای رزرو بلیط سینما.
				\item ویژگی‌ها:
				\begin{itemize}
					\item نمایش لیست فیلم‌ها و سالن‌های نمایش.
					\item رزرو صندلی‌های سینما برای فیلم‌های مختلف.
					\item پردازش پرداخت و ارسال تاییدیه به کاربران.
				\end{itemize}
				\item مثال:
				\begin{itemize}
					\item کاربر باید بتواند فیلم مورد نظر خود را انتخاب کرده و صندلی‌های خالی را مشاهده کند.
					\item پس از انتخاب صندلی، کاربر می‌تواند مبلغ بلیط را پرداخت کند و بلیط خود را دریافت کند.
				\end{itemize}
			\end{itemize}
			
			\item \textbf{برنامه ردیابی مخارج شخصی}
			\begin{itemize}
				\item هدف: ایجاد برنامه‌ای برای پیگیری مخارج شخصی.
				\item ویژگی‌ها:
				\begin{itemize}
					\item ثبت انواع هزینه‌ها مانند خوراک، حمل و نقل و مسکن.
					\item نمایش گزارش‌های ماهانه و سالانه.
					\item دسته‌بندی هزینه‌ها بر اساس نوع و تاریخ.
				\end{itemize}
				\item مثال:
				\begin{itemize}
					\item کاربر می‌تواند هزینه‌های روزانه خود را وارد کند و گزارش ماهانه‌ای از مخارج خود دریافت کند.
					\item سیستم می‌تواند به طور خودکار هزینه‌ها را دسته‌بندی کرده و درصد هر دسته را نشان دهد.
				\end{itemize}
			\end{itemize}
			
			\item \textbf{سیستم پرسش و پاسخ آنلاین}
			\begin{itemize}
				\item هدف: طراحی یک پلتفرم پرسش و پاسخ مشابه StackOverflow.
				\item ویژگی‌ها:
				\begin{itemize}
					\item کاربران قادر به ارسال سوالات و پاسخ‌ها باشند.
					\item قابلیت رای‌دهی به پاسخ‌ها و رتبه‌بندی آنها.
					\item امکان جستجوی سوالات و پاسخ‌ها.
				\end{itemize}
				\item مثال:
				\begin{itemize}
					\item کاربران می‌توانند سوالات خود را ارسال کرده و سایر کاربران به آنها پاسخ دهند.
					\item بهترین پاسخ‌ها بر اساس رای‌های کاربران در بالای صفحه نمایش داده می‌شود.
				\end{itemize}
			\end{itemize}
			
			\item \textbf{سیستم مدیریت انبار}
			\begin{itemize}
				\item هدف: طراحی سیستمی برای مدیریت موجودی انبار.
				\item ویژگی‌ها:
				\begin{itemize}
					\item ثبت ورود و خروج کالاها.
					\item پیگیری سطح موجودی و ارسال هشدار در صورت کمبود کالا.
					\item گزارش‌گیری از وضعیت انبار و میزان کالاها.
				\end{itemize}
				\item مثال:
				\begin{itemize}
					\item کاربر می‌تواند وضعیت موجودی هر کالا را مشاهده کرده و در صورت کمبود، سفارش جدید ثبت کند.
					\item سیستم باید گزارشی از کالاهای موجود و تاریخ‌های ورود و خروج آنها نمایش دهد.
				\end{itemize}
			\end{itemize}
			
			\item \textbf{سیستم شبیه‌سازی بازار سهام}
			\begin{itemize}
				\item هدف: شبیه‌سازی خرید و فروش سهام.
				\item ویژگی‌ها:
				\begin{itemize}
					\item نمایش قیمت‌های سهام و تغییرات آنها.
					\item امکان خرید و فروش سهام به‌طور شبیه‌سازی شده.
					\item گزارش‌گیری از سود و زیان کاربران.
				\end{itemize}
				\item مثال:
				\begin{itemize}
					\item کاربران می‌توانند سهام شرکت‌های مختلف را خریداری کرده و در زمان مناسب آن را بفروشند.
					\item سیستم باید تغییرات قیمت‌ها را به صورت تصادفی شبیه‌سازی کند و گزارشی از وضعیت مالی کاربران ارائه دهد.
				\end{itemize}
			\end{itemize}
			
			\item \textbf{برنامه تحلیل داده‌های آب و هوا}
			\begin{itemize}
				\item هدف: طراحی سیستمی برای نمایش و تحلیل داده‌های آب و هوای مناطق مختلف.
				\item ویژگی‌ها:
				\begin{itemize}
					\item دریافت اطلاعات آب و هوا از APIهای آنلاین.
					\item پیش‌بینی وضعیت آب و هوا بر اساس داده‌های تاریخی.
					\item نمایش دما، رطوبت، سرعت باد و پیش‌بینی‌های روزانه.
				\end{itemize}
				\item مثال:
				\begin{itemize}
					\item کاربر می‌تواند وضعیت آب و هوا را برای یک منطقه خاص مشاهده کرده و پیش‌بینی‌هایی برای روزهای آینده دریافت کند.
				\end{itemize}
			\end{itemize}
			
			\item \textbf{سیستم مدیریت پروژه تیمی}
			\begin{itemize}
				\item هدف: طراحی سیستمی برای مدیریت پروژه‌های تیمی.
				\item ویژگی‌ها:
				\begin{itemize}
					\item ایجاد و مدیریت وظایف پروژه.
					\item تخصیص وظایف به اعضای تیم.
					\item پیگیری وضعیت پیشرفت پروژه و ارسال یادآوری‌ها.
				\end{itemize}
				\item مثال:
				\begin{itemize}
					\item مدیر پروژه می‌تواند وظایف مختلف را ایجاد کند و به اعضای تیم اختصاص دهد.
					\item اعضای تیم باید بتوانند وضعیت پیشرفت خود را گزارش کرده و به کارهای محول شده رسیدگی کنند.
				\end{itemize}
			\end{itemize}
			
			\item \textbf{برنامه تحلیل داده‌های ورزشی}
			\begin{itemize}
				\item هدف: طراحی برنامه‌ای برای تحلیل داده‌های ورزشی.
				\item ویژگی‌ها:
				\begin{itemize}
					\item ثبت آمار بازیکنان و تیم‌ها.
					\item تحلیل نتایج مسابقات و پیش‌بینی نتایج آینده.
					\item نمایش آمار مقایسه‌ای بین بازیکنان و تیم‌ها.
				\end{itemize}
				\item مثال:
				\begin{itemize}
					\item کاربران می‌توانند آمار بازیکنان مانند تعداد گل‌ها، پاس‌ها و دقایق بازی را مشاهده کنند.
					\item سیستم می‌تواند پیش‌بینی‌هایی برای نتایج مسابقات آینده بر اساس داده‌های قبلی ارائه دهد.
				\end{itemize}
			\end{itemize}
			
		\end{enumerate}
		
	
	
\end{document}
